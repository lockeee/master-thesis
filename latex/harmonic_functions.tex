\section{SLTRs durch harmonische Funktionen}

Wir werden die Beweise zu den in diesem Abschnitt aufgestellten Präpositionen und Theoremen übergehen. Der interessierte Leser sei, sofern nicht anders angegeben auf \cite{af13} verwiesen. Im Verlauf des Abschnitts werdendes weiteren eine kurze Einführung in harmonische Funktionen auf planaren Graphen geben. Zum Einstieg formulieren wir eine weitere Definition um dann eine Beobachtung zu SLTRs festzuhalten.

\begin{definition}[Begrenzende Zykel und kombinatorisch konvexe Ecken]
Sei $G$ ein planer Graph mit Aufhängungen $\{a_1,a_2,a_3\}$ und einem FAA $\phi$ von G. Sei $H$ ein zusammenhängender Teilgraph von G und $\gamma=\gamma(H)$ der H umrandende Weg in G, also die Kanten und Knoten des äusseren Gebiets von H, wobei hier Knoten und Kanten mehrfach vorkommen können. Wir werden so erhaltene $\gamma$ als \textit{begrenzende Zykel} bezeichnen. $int(\gamma)$ sei die Menge aller Knoten, Kanten und Gebiete aus G die im inneren von $\gamma$ oder auf $\gamma$ liegen. Einen Knoten $v$ aus $\gamma$ bezeichnen wir als \textit{kombinatorisch konvexe Ecke} von $\gamma$ im Bezug auf $\phi$, falls gilt:
\begin{itemize}
\item [E1] $v$ ist eine Aufhängung, oder
\item [E2] $v$ ist nicht durch $\phi$ zugeordnet und es existiert eine Kante $e = (v,w)$ mit $e \notin int(\gamma)$, oder
\item [E3] $v$ ist einem Gebiet $f$ zugeordnet, $f \notin int(\gamma)$ und es existiert eine Kante $e = (v,w)$, sodass $e \notin int(\gamma)$.
\end{itemize}

\end{definition}

\begin{figure}[h]
	\centering
  \includegraphics[width=0.9\textwidth]{corner_def.png}
  \caption{Auf der linken Seite zwei Beispiele für \textit{begrenzende Zykel} und rechts für \textit{kombinatorisch konvexe Ecken} mit und ohne zugewiesenem Knoten.}
\end{figure}

Es lässt sich für SLTRs leicht zeigen, dass für jeden begrenzenden Zykel $\gamma$, der nicht von einem Pfad induziert wird, gilt, dass er mindestens drei kombinatorisch konvexe Ecken besitzt. Die folgende Präposition nach \cite[Prop 2.2, Prop 2.4]{af13} verallgemeinert diese Beobachtung.

\begin{proposition}\label{com_prop}
Sei $G$ ein planer Graph der eine SLTR $\Gamma$ zulässt. Sei weiter $\phi$ das von $\Gamma$ induzierte FAA und $H$ ein zusammenhängender Teilgraph von G. Falls $v$ eine geometrisch konvexe Ecke in $\Gamma$ ist, dann ist $v$ auch eine kombinatorisch konvexe Ecke hinsichtlich $\phi$. Somit gilt:
\begin{itemize}
\item [E4] Jeder begrenzende Zykel $\gamma$, der nicht von einem Pfad induziert wird, hat hinsichtlich $\phi$ mindestens drei kombinatorisch konvexe Ecken.
\end{itemize}

\end{proposition}

Proposition \ref{com_prop} liefert also eine notwendige Bedingung damit ein FAA von einer SLTR induziert sein kann. Dies ist sogar eine hinreichende Bedingung wie im Verlauf des Abschnittes in Theorem \ref{com_theo} gezeigt werden wird. Wir nennen ein FAA das [E4] erfüllt im Weiteren \textit{Gutes-FAA} oder kurz \textit{GFAA}. Aerts und Felsner zeigen, dass ein Gutes-FAA eine \textit{Kontaktfamilie von Pseudosegmenten} induziert die \textit{dehnbar} ist und sich somit geradlinig darstellen lässt.

\begin{definition}[Kontaktfamilie von Pseudosegmenten]

Eine \textit{Kontaktfamilie von Pseudosegmenten} ist eine Familie $\Sigma = \{c_i\}_i$ von einfachen Kurven $$c_i:[0,1] \to \mathbb{R}^2, \text{ mit } c_i(0) \neq c_i(1),$$ sodass alle Kurven $c_i,c_j$ mit $i \neq j$ maximal einen gemeinsamen Punkt haben. Dieser Punkt muss dann ein Endpunkt von mindestens einer der Kurven sein.

\end{definition}

Ein GFAA $\phi$ liefert eine Relation $\rho$ auf den Kanten von G. Zwei Kanten $(v,w),(v,u)$, beide adjazent zu $f$, stehen genau dann in Relation, falls $\phi(v)=f$, wenn also $(v,w)$ und $(v,u)$ auf der selben Seite von $f$ in der SLTR liegen. Der transitive Abschluss dieser Relation liefert eine Äquivalenzrelation $\rho$. Die Aquivalenzklassen von $\rho$ bilden eine Kontaktfamilie von Pseudosegmenten. Nennen wir die Äquivalenzklassen von $\rho$ Kurven, dann liefert F2, dass jeder Knoten nur im inneren von einer Kurve liegt und sich die Kurven nicht kreuzen. Weiter hat jede Kurve unterschiedliche Anfangs- und Endpunkte und kann sich nicht selbst berühren, da der resultierende begrenzende Zykel $\gamma$ nur eine beziehungsweise zwei kombinatorisch konvexe Ecken, was ein Widerspruch zu E4 wäre. Analog können zwei Kurven nicht ihre Anfangs- und Endpunkte teilen.\
Für eine von einem FAA $\phi$ induzierte Kontaktfamilie schreiben wir auch $\Sigma_{\phi}$.

\begin{figure}[h]
	\centering
  \includegraphics[width=0.9\textwidth]{pseudo_seg.png}
  \caption{Die Kanten von G als Kontaktfamilie von Pseudosegmenten induziert durch die Äquivalenzrelation. In rot und grün die beiden Äquivalenzklassen bzw. Kurven, die mehr als eine Kante beinhalten.}
\end{figure}

\begin{definition}
Sei $\Sigma$ ein Kontaktfamilie von Pseudosegmenten und $S\subseteq\Sigma$. Wir nennen einen Punkt $p\in S$ einen \textit{freien Punkt}, falls
\begin{itemize}
\item p ist ein Endpunkt eines Pseudosegmentes aus S.
\item p liegt nicht im Inneren eines Pseudosegmentes aus S.
\item p liegt am äusseren Rand von S.
\item p ist entweder eine Aufhängung von G, oder berührt ein Pseudosegment, welchen nicht zu S gehört.
\end{itemize} 
\end{definition}

\begin{lemma}\cite[Lemma 2.8]{af13}
Sei $\phi$ ein Gutes-FAA auf einem planen und intern 3-zusammenhängenden Graphen. Dann hat gilt: 
\begin{itemize}
\item [E5] jede Teilmenge $S \subseteq \Sigma_{\phi}$ mit $|S| \geq 2$ hat mindestens 3 freie Punkte.
\end{itemize}
\end{lemma}

Betrachten wir also einen planen, intern 3-zusammenhängenden Graphen $G$ mit Aufhängungen $\{a_1,a_2,a_3\}$ und einem GFAA $\phi$. Wenn wir die von $\phi$ induzierte Kontaktfamilie $\Sigma_{\phi}$ mit geradlinigen Segmenten darstellen können, dann haben wir eine zu $\phi$ passende SLTR für G gefunden. Für den Fall, dass eine solche Darstellung existiert, können wir für die resultierenden Koordinaten der Segmente und somit auch der Knoten von G Gleichungen aufstellen, die diese erfüllen müssten.

\subsection{Harmonische Funktionen auf planaren Graphen}

Das resultierende Gleichungssystem beinhaltet harmonische Funktionen, über die wir hier einen kurzen Überblick nach \cite{lov99} geben. Beginnen wir mit der Definition.

\begin{definition}[Harmonische Funktionen]
Sei $G=(V,E)$ ein planarer Graph und $S \supseteq V$. Eine Funktion $f:V \to \mathbb{R}$ nennen wir am Knoten $v \in V$ \textit{harmonsich}, falls gilt:
$$ \text{H1} & \frac{1}{d_v} \sum_{u \in N(v)}(f(u) - f(v)) = 0 & \forall v \in V \backslash S$$
Wir können H1 durch das hinzufügen einer nichtnegativen Gewichtsfunktion $\lambda:E\to\mathbb{R}_+$ verallgemeinern:
$$ \text{H2}\quad\frac{1}{d_v} \sum_{u \in N(v)}\lambda_{uv}(f(u) - f(v)) = 0 \qquad \forall v \in V \backslash S \qquad\qquad\qquad\qquad\qquad\qquad\quad$$
\end{definition}






Nehmen wir für den Moment an, dass wir eine Darstellung gefunden haben, dann gilt für jeden inneren Knoten $v$ auf einem Segment, dass er auf einer Gerade zwischen seinen beiden benachbarten Knoten $u,w$ auf dem Segment liegen muss. Diese Eigenschaft liefert 

