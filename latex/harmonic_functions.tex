\section{SLTRs durch harmonische Funktionen}\label{harmonic_approach}

Die Beweise zu den in diesem Kapitel aufgestellten Präpositionen und Theoremen werden ausgelassen. Sie befinden sich, wenn nicht anders angegeben, in \cite{af13} . Zum Einstieg eine weitere Definition, die es ermöglicht eine Beobachtung zu SLTRs festzuhalten.

\begin{definition}[Begrenzende Zykel und kombinatorisch konvexe Ecken]\label{def_ccc}
Sei $G$ ein planer Graph mit Aufhängungen $\{a_1,a_2,a_3\}$ und einem FAA $\phi$ von G. Sei $H$ ein zusammenhängender Teilgraph von G und $\gamma=\gamma(H)$, der H umrandende Weg in G. Bei $\gamma$ handelt es sich um die Kanten und Knoten des äusseren Gebiets von H. Knoten und Kanten können mehrfach vorkommen. Wir werden so erhaltene $\gamma$ als \textit{begrenzende Zykel} bezeichnen. $int(\gamma)$ sei die Menge aller Knoten, Kanten und Gebiete aus G die im Inneren von $\gamma$ oder auf $\gamma$ liegen. Einen Knoten $v$ aus $\gamma$ bezeichnen wir als \textit{kombinatorisch konvexe Ecke} von $\gamma$ im Bezug auf $\phi$, falls gilt:
\begin{itemize}
\item [E1] $v$ ist eine Aufhängung, oder
\item [E2] $v$ ist nicht durch $\phi$ zugeordnet und es existiert eine Kante $e = (v,w)$ mit $e \notin int(\gamma)$, oder
\item [E3] $v$ ist einem Gebiet $f$ zugeordnet, $f \notin int(\gamma)$ und es existiert eine Kante $e = (v,w)$, sodass $e \notin int(\gamma)$.
\end{itemize}

\end{definition}

\begin{figure}[h]
	\centering
  \includegraphics[width=0.9\textwidth]{corner_def.png}
  \caption{Auf der linken Seite zwei Beispiele für \textit{begrenzende Zykel} und rechts für \textit{kombinatorisch konvexe Ecken} mit und ohne zugewiesenem Knoten.}
\end{figure}

Es lässt sich für SLTRs leicht sehen, dass für jeden begrenzenden Zykel $\gamma$, der nicht von einem Pfad induziert wird, gilt, dass er mindestens drei kombinatorisch konvexe Ecken besitzt. Die folgende Präposition nach \cite[Prop 2.2, Prop 2.4]{af13} verallgemeinert diese Beobachtung.

\begin{proposition}\label{com_prop}
Sei $G$ ein planer Graph der eine SLTR $\Gamma$ zulässt. Sei weiter $\phi$ das von $\Gamma$ induzierte FAA und $H$ ein zusammenhängender Teilgraph von G. Falls $v$ eine geometrisch konvexe Ecke in $\Gamma$ ist, dann ist $v$ auch eine kombinatorisch konvexe Ecke hinsichtlich $\phi$. Somit gilt:
\begin{itemize}
\item [E4] Jeder begrenzende Zykel $\gamma$, der nicht von einem Pfad induziert wird, hat hinsichtlich $\phi$ mindestens drei kombinatorisch konvexe Ecken.
\end{itemize}

\end{proposition}

Proposition \ref{com_prop} liefert also eine notwendige Bedingung damit ein FAA von einer SLTR induziert sein kann. Dies ist sogar eine hinreichende Bedingung, wie im Verlauf des Kapitels in Theorem \ref{com_theo} gezeigt wird. Wir nennen ein FAA, das E4 erfüllt, im Weiteren \textit{Gutes-FAA} oder kurz \textit{GFAA}. Aerts und Felsner zeigen, dass ein Gutes-FAA eine \textit{Kontaktfamilie von Pseudosegmenten} induziert die \textit{dehnbar} ist und sich somit geradlinig darstellen lässt.

\begin{definition}[Kontaktfamilie von Pseudosegmenten]

Eine \textit{Kontaktfamilie von Pseudosegmenten} ist eine Familie $\Sigma = \{c_i\}_i$ von einfachen Kurven $$c_i:[0,1] \to \mathbb{R}^2, \text{ mit } c_i(0) \neq c_i(1),$$ sodass alle Kurven $c_i,c_j$ mit $i \neq j$ maximal einen gemeinsamen Punkt haben. Dieser Punkt muss dann ein Endpunkt von mindestens einer der Kurven sein.

\end{definition}

Ein GFAA $\phi$ liefert eine Relation $\rho$ auf den Kanten von G. Zwei Kanten $(v,w)$ und $(v,u)$, beide adjazent zu $f$, stehen genau dann in Relation, wenn $\phi(v)=f$. $(v,w)$ und $(v,u)$ müssen also auf der selben Seite  des Dreiecks $f$ in der SLTR liegen. Der transitive Abschluss dieser Relation liefert eine Äquivalenzrelation $\rho$. Die Aquivalenzklassen von $\rho$ bilden eine Kontaktfamilie von Pseudosegmenten. Nennen wir die Äquivalenzklassen von $\rho$ Kurven, dann gilt nach F2, dass jeder Knoten nur im Inneren von einer Kurve liegt und sich die Kurven nicht kreuzen. Weiter hat jede Kurve unterschiedliche Anfangs- und Endpunkte und kann sich nicht selbst berühren, da dann der resultierende begrenzende Zykel $\gamma$ nur eine beziehungsweise zwei kombinatorisch konvexe Ecken hätte, was ein Widerspruch zu E4 wäre. Analog können zwei Kurven nicht ihre Anfangs- und Endpunkte teilen.\
Für eine von einem FAA $\phi$ induzierte Kontaktfamilie schreiben wir auch $\Sigma_{\phi}$.

\begin{figure}[h]
	\centering
  \includegraphics[width=0.9\textwidth]{pseudo_seg.png}
  \caption{Die Kanten von G als Kontaktfamilie von Pseudosegmenten induziert durch die Äquivalenzrelation. In rot und grün die beiden Äquivalenzklassen bzw. Kurven, die mehr als eine Kante beinhalten.}
\end{figure}

\begin{definition}
Sei $\Sigma$ ein Kontaktfamilie von Pseudosegmenten und $S\subseteq\Sigma$. Wir nennen einen Punkt $p\in S$ einen \textit{freien Punkt}, falls er die folgenden Bedingungen erfüllt.
\begin{itemize}
\item p ist ein Endpunkt eines Pseudosegmentes aus S.
\item p liegt nicht im Inneren eines Pseudosegmentes aus S.
\item p liegt am äusseren Rand von S.
\item p ist entweder eine Aufhängung von G oder berührt ein Pseudosegment, welches nicht zu S gehört.
\end{itemize} 
\end{definition}

\begin{lemma}\cite[Lemma 2.8]{af13}\label{lemma_af13}
Sei $\phi$ ein Gutes-FAA auf einem planen und intern 3-zusammenhängenden Graphen. Dann gilt: 
\begin{itemize}
\item [E5] Jede Teilmenge $S \subseteq \Sigma_{\phi}$ mit $|S| \geq 2$ hat mindestens 3 freie Punkte.
\end{itemize}
\end{lemma}

Betrachte einen planen, intern 3-zusammenhängenden Graphen $G$ mit Aufhängungen $\{a_1,a_2,a_3\}$ und einem GFAA $\phi$. Wenn die von $\phi$ induzierte Kontaktfamilie $\Sigma_{\phi}$ mit geradlinigen Segmenten darstellbar ist, dann ist diese Darstellung eine zu $\phi$ passende SLTR für $G$. Für den Fall, dass eine solche Darstellung $f:G\to\mathbb{R}^2$ existiert, können für die Koordinaten der Segmente und somit auch der Knoten $v$ von $G$ Gleichungen aufgestellt werden. Die Positionen der Knoten $v$ in der Einbettung $f(v)$ müssen diese Gleichungen erfüllen. Das resultierende Gleichungssystem beinhaltet harmonische Funktionen. Zu diesen folgt ein kurzer Überblick.

\subsection{Harmonische Funktionen auf planaren Graphen}

Die Theorie zu (diskreten) harmonischen Funktionen auf planaren Graphen und ihre Anwendung werden in \cite{lov99} ausführlich behandelt. Es handelt sich um eine Diskretisierung von allgemeinen harmonischen Funktionen, also glatten Funktionen $f:G\subseteq \mathbb{R}^n \to \mathbb{R}$, mit $\Delta f = 0$. Für diese Funktionen gilt, dass der Funktionswert an einem Punkt $x$, dem Durchschnitt der Funktionswerte auf einem Ball um $x$ entspricht. 

Dies führt zu der folgenden Definition.

\begin{definition}[Harmonische Funktionen]
Sei $G=(V,E)$ ein planarer zusammenhängender Graph und $S \subseteq V$. Eine Funktion $g:V \to \mathbb{R}$ nennen wir am Knoten $v \in V$ \textit{harmonisch}, falls gilt:
$$ \text{H1} \quad \frac{1}{deg(v)} \sum_{u \in N(v)}(g(u) - g(v)) = 0 \quad \forall v \in V \backslash S \qquad\qquad\qquad\qquad\qquad\qquad\quad\:\,\:$$
Wir können H1 durch das hinzufügen einer nichtnegativen Gewichtsfunktion $\lambda:E\to\mathbb{R}_+$ verallgemeinern. Es gilt $\lambda((v,w)) = \lambda_{vw}$.
$$ \text{H2}\quad\frac{1}{deg(v)} \sum_{u \in N(v)}\lambda_{uv}(g(u) - g(v)) = 0 \quad \forall v \in V \backslash S \qquad\qquad\qquad\qquad\qquad\qquad$$
Ein Knoten für den $g$ nicht harmonisch ist, nennt man \textit{Pol}.
\end{definition}

\begin{theorem}\cite[Theorem 3.1.2]{lov99}\label{harmonic_uni}
Für jede nichtleere Teilmenge $S \subseteq V$ und jede Funktion $g_S:S\to\mathbb{R}$ existiert genau eine Funktion $g:V\to\mathbb{R}$, die $f_S$ auf $V$ fortsetzt, sodass $g$ in jedem Knoten $v\in V \backslash S$ harmonisch ist. Wir nennen sie die \textup{harmonische Fortsetzung}.
\end{theorem}

Ein bekanntes Resultat, dass sich in Form harmonischer Funktionen darstellen lässt, ist Tuttes \textit{rubber-band-representation} aus \cite{tutte63}, die konvexe Zeichnungen für planare Graphen liefert. Man stelle sich einen planaren Graphen vor, bei dem jede Kante durch ein idealisiertes Gummiband\footnote{Die Gummibänder müssen das Hook'sche Gesetzt erfüllen, sodass eine Streckung auf Länge $l$ genau Kraft $l$ benötigt.} ersetzt wird. Wähle nun ein äusseres Gebiet und fixiere die Knoten $S\subseteq V$, die in an diesem Gebiet liegen, in zyklischer Reihenfolge und in gleichen Abständen auf einem Kreis in der Ebene. Dies definiert $f_S:S \to \mathbb{R}^2$. Die restlichen Knoten werden nun von den Bändern in eine neue Position gezogen. Das resultierende Gleichgewicht, das genau dann entsteht, wenn H1 erfüllt ist, entspricht der harmonischen Fortsetzung von $f_S$  auf $V$, wobei $f(v)$ genau der Position von $v$ in der resultierenden Einbettung entspricht und $S$ die Menge der Pole von $f$ ist. Wir können die Kanten zusätzlich noch mit nicht negativen Gewichten $\lambda_{vw}$, versehen um die Einbettung zu verändern. Das folgende Theorem ist das Hauptresultat aus \cite{tutte63}.

\begin{theorem}\label{theo_rubber}
Sei $G$ ein planarer Graph, dann ist eine \textit{Gummiband-Representation (rubber-band-representation)} von $G$ eine planare Einbettung in der Ebene.
\end{theorem}

Die Theorie zu harmonischen Funktionen lässt sich auf SLTRs anwenden. Nehme für den Moment an, dass es existiert eine geradlinige Darstellung der Pseudosegmente. Wir haben also eine geradlinige Einbettung $f$ der von $\phi$ induzierten Segmente. Dann gilt für jeden Knoten $v$ im Inneren eines Segmentes, also für jeden zugewiesenen Knoten, dass er auf einer Gerade zwischen seinen beiden benachbarten Knoten $u,w$ auf dem Segment liegen muss. Diese Eigenschaft liefert
\begin{equation}\label{harm_1}
f(v) = \lambda_v f(u) + (1-\lambda_v)f(w) \text{, mit } \lambda_v \in (0,1).
\end{equation}
Für die nicht zugewiesenen Knoten aus $G$ muss in einer SLTR gelten, dass sie sich in der konvexen Hülle ihrer Nachbarn befinden. Wir bilden einen (gewichteten) Schwerpunkt und erhalten
\begin{equation}\label{harm_2}
f(v) = \sum_{u \in N(v)} \lambda_{uv} f(u) \text{, mit }  \sum_{u \in N(v)}\lambda_{uv} = 1 \text{ und } \lambda_{uv} \geq 0.
\end{equation}

Somit erfüllt die so gegebene Funktion $f:V\to\mathbb{R}^2$ mit einem passend gewählten $\lambda$ wegen (\ref{harm_1}) und (\ref{harm_2}) in beiden Komponenten H2. Es handelt sich somit bei $f_1$ und $f_2$ um harmonische Funktionen, mit den Polen $\{a_1,a_2,a_3\}$. Nach Theorem \ref{harmonic_uni}, existiert für jede den Beschränkungen entsprechende Wahl von $\lambda$ jeweils genau eine Funktion $f_1,f_2$, welche die Gleichungen erfüllen.\\
Dies führt uns zum Hauptresultat aus \cite[Theorem 2.10]{af13}:

\begin{theorem}\label{com_theo}
Sei $G$ ein intern 3-zusammenhängender, planarer Graph und $\Sigma$ eine Familie von Pseudosegmenten, induziert von einem FAA, sodass jede Teilfamilie $S \subset \Sigma$ entweder mindestens drei freie Punkte hat, oder maximal ein Element enthält. Die eindeutige Lösung des aus $\Sigma$ folgenden Gleichungssystems ist eine SLTR.
\end{theorem}

\begin{remark}
Dies bedeutet, dass die weiter oben in Lemma \ref{lemma_af13} festgehaltene notwendige Bedingung auch eine hinreichende ist. Falls wir schon ein Gutes-FAA gefunden haben, dann können wir mit Hilfe des obrigen Ansatzes auch eine Einbettung in der Ebene erhalten. Jedoch gibt es Graphen mit polynominell vielen FAA und es dauert polynominell lange, um E4 zu überprüfen. Wir erreichen also auf diesem Weg keinen Algorithmus in $\mathcal{P}$.\\
Aerts und Felsner werfen am Ende des Papers die Frage nach einer \textit{guten} Wahl von $\lambda$ auf und wie dies die resultierenden Einbettungen beeinflussen kann. Dieses Thema wir in Kapitel \ref{the_program} angegangen.
\end{remark}


