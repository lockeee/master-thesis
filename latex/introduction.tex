\chapter{Einleitung}\label{intro}

Wir werden uns in dieser Arbeit hauptsächlich mit einfachen planeren Graphen beschäftigen, also solchen die keine Mehrfachkanten und Schleifen besitzen und für die kreuzungsfreie Zeichnungen, beziehungsweise Einbettungen, in der Ebene existieren. Sei $G = (V,E)$ ein Graph bestehend aus der Menge der Knoten $V$ und Kanten $E \subseteq ( \,V \times V ) \,$. Eine Kante $uv$ verbindet die beiden Knoten $u$ und $v$. Einen planeren Graphen zusammen mit einer möglichen kreuzungsfreien Einbettung in der Ebene bezeichnen wir als \textit{planen Graphen}. Sei Für einen planaren Graphen können wir, zusätzlich zu den Knoten und Kanten, auch die Menge der Gebiete (engl. faces) $F$ betrachten. Bei einem planen Graph wird das unbeschränkte als das \textit{äussere} Gebiet definiert. Für die weiteren Betrachtungen macht es oft Sinn drei Knoten $a_1,a_2,a_3$ im äusseren Gebiet gesondert zu betrachten und diese die \textit{Aufhängungen} von $G$ zu nennen.\\

Planare Graphen haben, durch die Existenz kreuzungsfreier Einbettungen, in gewissem Sinne besonders schöne Zeichnungen und so ist einer der Fragen mit der sich schon viele Mathematiker auseinander gesetzt haben: \textit{"How to draw a Graph?"}\cite{tutte63}\\

Bei topologische Zeichnung eines planaren Graphen werden die Kanten als Kurven dargestellt die sich nur in den Knoten treffen. In den Fünfzigern wurde unter anderem von István Fáry gezeigt, dass für jeden planaren Graphen mit einem beliebigen äusseren Gebiet eine geradlinige Zeichnung existiert. \cite{fary48}

\begin{figure}
	\centering
  \includegraphics[width=0.9\textwidth]{topo_straight_convex.png}
	\caption{Planarer Graph mit einer topologischen, einer geradlinigen und einer konvexen Zeichnung.}
	\label{cut_figure}
\end{figure}

\begin{definition}[intern zusammenhängend]\label{int_3_con}
Ein Graph $G$ ist zusammenhängen falls für alle Knoten $u,v$ ein Pfad von $u$ nach $v$ exisitert. $G$ ist \textit{k-zusammenhängend}, falls er nach der Entfernung von $k-1$ beliebigen Knoten weiterhin zusammanhängend ist.\\
Sei $G$ plan mit den Aufhängungen $a_1,s_2,a_3$, weiter sei $a_\infty$ ein zusätzlicher Knoten im äusseren Gebiet. Dann ist $G$ \textit{intern k-zusammenhängend}, falls $G \cup \{ a_1a_\infty,a_2a_\infty,a_as_\infty \}$ k-zusammenhängend ist. 
\end{definition}

In den Siebzigern betrachtete William Thomas Tutte die Unterklasse der drei-zusammenhängenden planaren Graphen und zeigte, dass für diese nicht nur geradlinige, sondern sogar \textit{konvexe} Zeichnungen existieren, bei denen alle Gebiete die konvexe Polygone umranden. \cite{tutte63}

\section{Geradlinige Dreiecks Darstellungen (SLTRs)}

Ausgehend von den konvexen Einbettungen nach Tutte, kann man sich die Frage stellen, unter welchen Vorraussetzungen wir einen planaren Graphen so zeichnen können, dass alle Gebiete Dreiecke umranden. Die Formalisierung dieser Darstellung und erste Folgerungen folgen Nieke Aerts und Stefan Felsner \cite{af13,af15}.

\begin{definition}[SLTR]\label{defsltr}
Eine Zeichnung eines planen Graphen $G$ wird \textit{geradlinige Dreiecks Darstellung}, im weiteren kurz \textit{SLTR} (für die englische Bezeichnung \textit{staight line triangle representation}), genannt falls gilt:
\begin{itemize}
\item[S1] Alle Kanten sind Segmente von Geraden.
\item[S2] Alle Gebiete, inklusive dem Äusseren, sind nicht degenerierte Dreiecke.
\end{itemize}
\end{definition}

\begin{figure}[h]
	\centering
  \includegraphics[width=0.9\textwidth]{sltr-example.png}
	\caption{Links einer der beiden 3-zusammenhängenden Graphen auf acht Knoten ohne SLTR und rechts ein Graph mit einer möglichen SLTR.}
\end{figure}

Um die Problemstellung greifbarer zu machen kann man plane Graphen zusammen mit den Aufhängungen $\{a_1,a_2,a_3\}$ betrachten, wobei $\{a_1,a_2,a_3\}$ hier die designierten Ecken des äusseren Gebietes einer möglichen SLTR sind. Einen Graphen zusammen mit einem äusseren Gebiet bzw. festen Aufhängungen als Paar zu behanden ist sinnvoll, weil planare Graphen existieren, von denen manche Einbettungen, SLTRs zulassen, andere jedoch nicht, so wie in Abbildung \ref{10_example} zu sehen. Zumindest für 3-zusammenhängende planare Graphen ist die topologische Einbettung nach der Auswahl der Aufhängungen eindeutig.

\begin{proposition}\cite[Proposition 1.2]{af13}
Sei $G$ ein planer Graph mit den Aufhängungen $\{a_1,a_2,a_3\}$ als äussere Ecken einer SLTR. Weiter gebe es keine inneren Knoten $v$ mit $deg(v) < 3$. Dann ist $G$ intern-3-zusammenhängend.
\end{proposition}

\begin{remark}
Für innere Knoten von Grad 2 in einer SLTR müssen beide angrenzenden Winkel gerade sein. Somit kann man diese Knoten durch eine gerade Kante zwischen ihren Nachbarn ersetzen und den resultierenden Graphen betrachten. Wir werden somit von nun an nur intern-3-zusammenhängende Graphen mit Aufhängungen betrachten, da alle anderen Graphen, die eine SLTR zulassen, auf diese reduziert werden können.
\end{remark}

\begin{figure}[h]
	\centering
  \includegraphics[scale=0.1]{10_example.png}
	\caption{Der kleinste 3-zusammenhängende kombinatorische Graph mit einer Wahl der Aufhängungen die eine SLTR zulässt und einer Auswahl ohne SLTR.}
	\label{10_example}
\end{figure}

Zu den Fragen, welche notwendigen und hinreichenden Bedingungen es für die Existenz von SLTRs gelten und  welche algorithmischen Ansätze man bei der Suche nach einer spezifischen Darstellung verfolgen kann, haben Aerts und Felsner in \cite{af13}, \cite{af13h} und \cite{af15} schon einige Antworten geliefert. Die nächsten zwei Kapitel, werden sich damit beschäftigen. Zuvor müssen in diesem Kapitel noch ein paar notwendige Konzepte eingeführt werden.


\section{Schnyder Woods}\label{sw}
Betrachten wir einen planeren Graphen mit Aufhängungen $a_1,a_2,a_3$. Anschaulich handelt es sich bei einem Schnyder Wald um drei Aufspannende Bäume $T_1,T_2,T_3$. Jeder der $T_i$ ist hin zu seiner Wurzel $a_i$ gerichtet und Kanten können von zwei der drei Bäume gleichzeitig genutzt werden.

Schnyder Wälder, im weiteren \textit{Schnyder Woods}, wurden zuerst von Walter Schnyder eingeführt. Sie dienten zur Betrachtung der Ordnungs-Dimension planarer Graphen, als eine Färbung und Orientierung auf den inneren Kanten einer Triangulierung \cite{schnyder89}. In einem weiteren Resultat dienten sie zur Erlangung einer planaren Einbettung auf einem $(n-2)\times(n-2)$ Gitter \cite{schnyder90}.

Wir wollen hier die Verallgemeinerung auf 3-zusammenhängende plane Graphen durch Felsner \cite{felsner01} und die zu ihnen in Bijektion stehenden Schnyder Labelings einführen. Wir orientierten uns an \cite{felsner04}. Für den Rest dieses Kapitels sei $G$, wenn nicht weiter spezifiziert, ein 3-zusammenhängenden planer Graph mit Aufhängungen $\{a_1,a_2,a_3\}$.

\begin{definition}[Schnyder Woods]\label{def_sw}
Ein Schnyder Wood ist eine Orientierung und Beschriftung der Kanten von $G$ mit den Labeln 1, 2 und 3 (alternativ wird hier auch oft rot, grün und blau genutzt)\footnote{Es wird davon ausgegangen, dass die Label zyklisch sortiert sind, sodass $i+1$ und $i-1$ immer definiert sind.}, unter Berücksichtigung der folgenden Regeln:
\begin{itemize}
\item[W1] Jede Kante ist entweder in eine oder zwei Richtungen orientiert. Falls sie bigerichtet ist haben beide Richtungen unterschiedliche Label.
\item[W2] An jeder Aufhängung  $a_i$ existiert eine nach aussen gerichtete Kante ohne Endpunkt mit Label i.  
\item[W3] Jeder Knoten $v$ hat hat Ausgangsgrad eins in jedem Label. Um $v$ existieren im Uhrzeigersinn eine Auskante mit Label 1, null oder mehr eingehende Kanten mit Label 3, eine Auskante mit Label 2, null oder mehr  eingehende Kanten mit Label 1, eine Auskante mit Label 2 und null oder mehr  eingehende Kanten mit Label 2.
\item[W4] Es existiert kein inneres Gebiet mit einem gerichteten Zykel in einer Farbe als Rand.
\end{itemize}
\end{definition}

\begin{figure}[h]
	\centering
  \includegraphics[width=0.8\textwidth]{schnyder_wood_def.png}
\end{figure}

Analog zu den Schnyder Woods, kann man Schnyder Labelings definieren, die zu diesen in Bijektion stehen. Hier betrachten wir nicht zuerst die Kanten eines planaren Graphen sondern die Winkel an den Knoten.

\begin{definition}[Schnyder Labeling]\label{def_sl}
Ein Schnyder Labeling ist eine Beschriftung der Winkel von $G$ mit den Labeln 1, 2 und 3 (oder rot, grün und blau) unter Berücksichtigung der folgenden Regeln:
\begin{itemize}
\item[L1] Um jedes innere Gebiet bilden die Label im Uhrzeigersinn nichtleere Intervalle von 1en, 2en und 3en. Am äusseren Gebiet gilt dies gegen den Uhrzeigersinn.
\item[L2] Um jeden inneren Knoten bilden die Label im Uhrzeigersinn nichtleere Intervalle von 1en, 2en und 3en.
\item[L3] An Aufhängung $a_i$ haben äusseren Winkel die Label i-1 und i+1 im Uhrzeigersinn mit der halben Auskante dazwischen und die inneren Winkel das Label i.
\end{itemize} 
\begin{figure}[h]
	\centering
  \includegraphics[width=0.8\textwidth]{schnyder_label_def.png}
\end{figure}
\end{definition}

In Abbildung \ref{schnyder_bij} wird eine Verbindung zwischen Schnyder Woods und Schnyder Labelings illustriert. Das nächste Lemma folgt aus L1 und L2.

\begin{lemma}\label{lem_sl}
Sei G ein planer, intern-3-zusammenhängender Graph mit den Aufhängungen $a_1,a_2,a_3$ und einem Schnyder Labeling. Dann beinhalten die vier Winkel entgegen dem Uhrzeigersinn an jeder Kante die Label 1, 2 und 3. Somit hat jede Kante einen der beiden Typen in Abbildung \ref{schnyder_bij}.
\end{lemma}

\begin{figure}[h]
	\centering
  \includegraphics[width=0.7\textwidth]{schnyder_bij.png}
	\caption{Bijektion zwischen Schnyder Wood auf der rechten und Schnyder Labeling auf der linken Seite.}
	\label{schnyder_bij}
\end{figure}

Wenn wir uns auf intern-3-zusammenhängende planare Graphen beschränken, dann ist die dargestellte Abbildung nach \cite[Theorem 2.3]{felsner04} eine Bijektion. Zu jedem Schnyder Labeling gehört also genau ein Schnyder Wood und anders herum. Dies macht es möglich, wenn sinnvoll, zwischen den beiden Strukturen hin und her zu wechseln. So kann es auch im Verlauf dieser Arbeit vorkommen, dass wir vom einen schreiben, aber implizit Eigenschaften des andern meinen.\\

Es existieren einige Anwendungen von Schnyder Woods im Bezug auf Einbettungen. Wie schon erwähnt bezieht sich eines der ersten Resultate auf die konvexe Einbettung auf einem Gitter. Eine Verbesserung des in \cite{schnyder90} erreichten ist das im Folgenden skizzierte \textit{face-counting} \cite{felsner01}. Betrachte $G$ mit einem Schnyder Wood $T_1,T_2,T_3$. Nach \cite[Korollar 2.5]{felsner04} handelt es sich bei den $T_i$ um gerichtete Bäume mit Wurzeln in $a_i$. Von jedem Knoten $v$ aus existierten also eindeutige Pfade $P_i(v)$ zu den Aufhängungen $a_i$. Die Pfade von $v$ zu den Aufhängungen treffen sich nach \cite[Lemma 2.4]{felsner04} nur in $v$. Wir erhalten also zu jedem Knoten $v$ drei Regionen $R_i$ , die jeweils von den Pfaden $P_{i-i}(v)$ und $P_{i+1}(v)$ und dem äusseren Gebiet eingegrenzt werden. In jeder dieser Regionen können wir nun die eingeschlossenen Gebiete von $G$ zählen. Durch das Zählen der Gebiete in den Regionen zu $v$ lässt sich eine konvexe Zeichnung von $G$ erzeugen.

Hierzu ordnet man jedem Knoten $v$ seien Gebiets Vektor $(v_1,v_2,v_3)$ zu, wobei $v_i$ die Anzahl der inneren Gebiete in $R_i(v)$ beschreibt. Nun gilt für jeden Knoten $v_1+v_2+v_3 = |F|-1$. Seien $\alpha_1 = (0,1),\alpha_2 = (1,0)$ und $\alpha_3 = (0,0)$ die äusseren Ecken unserer Zeichnung. Sie entsprechen ebenfalls den Bildern der Aufhängungen von $G$. Die Position der inneren Knoten ergibt sich nun durch die Funktion 
$$\mu: V \to \mathbb{R}^2,v\mapsto v_1\alpha_1 + v_2\alpha_2+v_3\alpha_3.$$ 

Nach \cite[Theorem 2.7]{felsner04} ist die mit diesen Koordinaten erzeugte Zeichnung planar und konvex und passt auf ein $(|F|-1)\times(|F|-1)$-Gitter. Sie hat noch eine weitere Eigenschaft die später von Nutzen sein wird und in Abbildung \ref{face_counting} dargestellt ist.

\begin{figure}
	\centering
  \includegraphics[width=0.8\textwidth]{face_counting.png}
	\caption{Eine Schnyder Wood und die durch \textit{face counting} erhaltene Einbettung. Die eingefärbten Gebiete sind die Regionen die den Gebietsvektor $(v_1,v_2,v_3)$ ergeben. In der Mitte ist W5 illustriert.}
	\label{face_counting}
\end{figure}

\begin{itemize}
\item [W5] Die Knoten eines inneren Gebietes werden auf die Seiten eines Dreiecks mit den Seiten $c_i(\alpha_{i-1}-\alpha_{i+1})$ mit passenden Konstanten $c_i$ abgebildet. Im inneren dieses Dreiecks befinden sich keine Knoten und die Winkel des Gebietes auf der Seite $c_i(\alpha_{i-1}-\alpha_{i+1})$ haben Label $i$ im Schnyder Labeling.
\end{itemize}



