\chapter{Einleitung}\label{intro}

Wir werden uns in dieser Arbeit hauptsächlich mit einfachen planeren Graphen beschäftigen, also solchen die keine Mehrfachkanten und Schleifen besitzen und für die kreuzungsfreie Zeichnungen, beziehungsweise Einbettungen, in der Ebene existieren. Sei $G = (V,E)$ ein Graph bestehend aus der Menge der Knoten $V$ und Kanten $E \subseteq ( \,V \times V ) \,$. Eine Kante $uv$ verbindet die beiden Knoten $u$ und $v$. Einen planeren Graphen zusammen mit einer möglichen kreuzungsfreien Einbettung in der Ebene bezeichnen wir als \textit{planen Graphen}. Sei Für einen planaren Graphen können wir, zusätzlich zu den Knoten und Kanten, auch die Menge der Gebiete (engl. faces) $F$ betrachten. Bei einem planen Graph wird das unbeschränkte als das \textit{äussere} Gebiet definiert. Für die weiteren Betrachtungen macht es oft Sinn drei Knoten $a_1,a_2,a_3$ im äusseren Gebiet gesondert zu betrachten und diese die \textit{Aufhängungen} von $G$ zu nennen.\\

