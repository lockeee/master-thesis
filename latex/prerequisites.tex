\chapter{Grundlagen}\label{pre}

Diese Arbeit wird sich mit einfachen planaren Graphen beschäftigen, also solchen die keine Mehrfachkanten und Schleifen besitzen und für die kreuzungsfreie Zeichnungen, beziehungsweise Einbettungen, in der Ebene existieren. Sei $G = (V,E)$ ein Graph bestehend aus der Menge der Knoten $V$ und Kanten $E \subseteq ( \,V \times V ) \,$. Eine Kante $(u,v)$ verbindet die beiden Knoten $u$ und $v$. Ein Pfad von $u$ nach $v$ ist eine Folge von Kanten, die $u$ und $v$ verbindet. Mit dem Grad $deg(v)$ eines Knoten meinen wir die Anzahl der adjazenten Kanten.\ 

Einen planaren Graphen zusammen mit einer möglichen kreuzungsfreien Einbettung in der Ebene bezeichnen wir als \textit{planen Graphen}. Für einen planen Graphen können wir, zusätzlich zu den Knoten und Kanten, auch die Menge der Gebiete (engl. faces) $F$ betrachten, die durch die Kanten und Knoten begrenzten Regionen in der Ebene, wobei wird das unbeschränkte als das \textit{äussere} Gebiet bezeichnen. Für die weiteren Betrachtungen macht es oft Sinn drei Knoten $\{a_1,a_2,a_3\}$ die das äusseren Gebiet berühren gesondert zu betrachten. Wir nennen sie die \textit{Aufhängungen} von $G$. \\

Planare Graphen haben durch die Existenz kreuzungsfreier Einbettungen in gewissem Sinne besonders schöne Zeichnungen. So ist einer der Fragen, mit der sich schon viele Mathematiker*innen auseinander gesetzt haben und auf die auch in dieser Arbeit eingegangen wird: \textit{"How to draw a Graph?"}\cite{tutte63}

Betrachten wir unterschiedliche Arten von Einbettungen. Bei einer topologischen Zeichnung eines planaren Graphen werden die Kanten als Kurven dargestellt, die sich nur in den Knoten treffen. Wir können diese Kurven auch als Geraden zeichnen. In den Fünfzigern wurde unter anderem von István Fáry gezeigt, dass für jeden planaren Graphen und für jede Wahl eines äusseren Gebietes eine geradlinige und kreuzungsfreie Einbettung existiert \cite{fary48}.

Um eine Unterklasse der planaren Graphen genauer zu betrachten brauchen wir folgende Definition.

\begin{definition}[intern zusammenhängend]\label{int_3_con}
Ein Graph $G$ ist zusammenhängend falls für alle Knoten $u,v$ ein Pfad von $u$ nach $v$ exisitert. $G$ ist \textit{k-zusammenhängend}, falls er nach der Entfernung von $k-1$ beliebigen Knoten weiterhin zusammanhängend ist.\\
Sei $G$ plan mit den Aufhängungen $\{a_1,a_2,a_3\}$, weiter sei $a_\infty$ ein zusätzlicher Knoten eingefügt im äusseren Gebiet. Dann ist $G$ \textit{intern k-zusammenhängend}, falls $G+v_\infty\coloneqq(V\cup\{v_\infty \},E\cup \{(a_1,a_\infty),(a_2,a_\infty),(a_3,a_\infty)\})$ k-zusammenhängend ist. 
\end{definition}

In den Siebzigern betrachtete William Thomas Tutte die Unterklasse der 3-zusam-menhängenden planaren Graphen und zeigte, dass für diese nicht nur geradlinige, sondern \textit{konvexe} Zeichnungen existieren. Bei einer konvexen Einbettung entsprechen die Kantenfogen, die ein Gebiet einschließen, den Randkurven von konvexen Polygonen \cite{tutte63}.

Diese Arbeit wird sich mit einer speziellen Form der konvexen Einbettung von planaren Graphen beschäftigen.

\begin{figure}
	\centering
  \includegraphics[width=0.9\textwidth]{topo_straight_convex.png}
	\caption{Planarer Graph mit einer topologischen, einer geradlinigen und einer konvexen Zeichnung.}
	\label{topo_straight_convex}
\end{figure}

\section{Geradlinige Dreiecks Darstellungen (SLTRs)}

Ausgehend von den konvexen Einbettungen nach Tutte, kann man sich die Frage stellen, unter welchen Vorraussetzungen wir einen planaren Graphen so zeichnen können, dass alle Gebiete Dreiecke umranden. Die Formalisierung dieser Darstellung und erste Folgerungen folgen Nieke Aerts und Stefan Felsner \cite{af13,af15}.

\begin{definition}[SLTR]\label{defsltr}
Eine Zeichnung eines planen Graphen $G$ wird \textit{geradlinige Dreiecks Darstellung}, im weiteren kurz \textit{SLTR} (für die englische Bezeichnung \textit{staight line triangle representation}), genannt falls gilt:
\begin{itemize}
\item[S1] Alle Kanten sind Segmente von Geraden.
\item[S2] Alle Gebiete, inklusive dem Äusseren, sind nicht degenerierte Dreiecke.
\end{itemize}
\end{definition}

\begin{figure}[h]
	\centering
  \includegraphics[width=0.9\textwidth]{sltr-example.png}
	\caption{Links einer der beiden 3-zusammenhängenden Graphen auf acht Knoten ohne SLTR und rechts ein Graph mit einer möglichen SLTR.}
\end{figure}

Um die Problemstellung greifbarer zu machen kann man plane Graphen zusammen mit den Aufhängungen $\{a_1,a_2,a_3\}$ betrachten, wobei $\{a_1,a_2,a_3\}$ hier die designierten Ecken des äusseren Gebietes einer möglichen SLTR sind. Einen Graphen zusammen mit einem äusseren Gebiet bzw. festen Aufhängungen als Paar zu behanden ist sinnvoll, weil planare Graphen existieren, von denen manche Einbettungen, SLTRs zulassen, andere jedoch nicht, so wie in Abbildung \ref{10_example} zu sehen. Zumindest für 3-zusammenhängende planare Graphen ist die topologische Einbettung nach der Auswahl der Aufhängungen eindeutig.

\begin{proposition}\cite[Proposition 1.2]{af13}
Sei $G$ ein planer Graph mit den Aufhängungen $\{a_1,a_2,a_3\}$ als äussere Ecken einer SLTR. Weiter gebe es keine inneren Knoten $v$ mit $deg(v) < 3$. Dann ist $G$ intern-3-zusammenhängend.
\end{proposition}

\begin{remark}
Für innere Knoten von Grad 2 in einer SLTR müssen beide angrenzenden Winkel gerade sein. Somit kann man diese Knoten durch eine gerade Kante zwischen ihren Nachbarn ersetzen und den resultierenden Graphen betrachten. Wir werden somit von nun an nur intern-3-zusammenhängende Graphen mit Aufhängungen betrachten, da alle anderen Graphen, die eine SLTR zulassen, auf diese reduziert werden können.
\end{remark}

\begin{figure}[h]
	\centering
  \includegraphics[scale=0.1]{10_example.png}
	\caption{Der kleinste 3-zusammenhängende kombinatorische Graph mit einer Wahl der Aufhängungen die eine SLTR zulässt und einer Auswahl ohne SLTR.}
	\label{10_example}
\end{figure}

Zu den Fragen, welche notwendigen und hinreichenden Bedingungen es für die Existenz von SLTRs gelten und  welche algorithmischen Ansätze man bei der Suche nach einer spezifischen Darstellung verfolgen kann, haben Aerts und Felsner in \cite{af13}, \cite{af13h} und \cite{af15} schon einige Antworten geliefert. Die nächsten zwei Kapitel, werden sich damit beschäftigen. Zuvor müssen in diesem Kapitel noch ein paar notwendige Konzepte eingeführt werden.

\section{Schnyder Woods}\label{sw}
Betrachten wir einen planeren Graphen mit Aufhängungen $a_1,a_2,a_3$. Anschaulich handelt es sich bei einem Schnyder Wald um drei Aufspannende Bäume $T_1,T_2,T_3$. Jeder der $T_i$ ist hin zu seiner Wurzel $a_i$ gerichtet und Kanten können von zwei der drei Bäume gleichzeitig genutzt werden.

Schnyder Wälder, im weiteren \textit{Schnyder Woods}, wurden zuerst von Walter Schnyder eingeführt. Sie dienten zur Betrachtung der Ordnungs-Dimension planarer Graphen, als eine Färbung und Orientierung auf den inneren Kanten einer Triangulierung \cite{schnyder89}. In einem weiteren Resultat dienten sie zur Erlangung einer planaren Einbettung auf einem $(n-2)\times(n-2)$ Gitter \cite{schnyder90}.

Wir wollen hier die Verallgemeinerung auf 3-zusammenhängende plane Graphen durch Felsner \cite{felsner01} und die zu ihnen in Bijektion stehenden Schnyder Labelings einführen. Wir orientierten uns an \cite{felsner04}. Für den Rest dieses Kapitels sei $G$, wenn nicht weiter spezifiziert, ein 3-zusammenhängenden planer Graph mit Aufhängungen $\{a_1,a_2,a_3\}$.

\begin{definition}[Schnyder Woods]\label{def_sw}
Ein Schnyder Wood ist eine Orientierung und Beschriftung der Kanten von $G$ mit den Labeln 1, 2 und 3 (alternativ wird hier auch oft rot, grün und blau genutzt)\footnote{Es wird davon ausgegangen, dass die Label zyklisch sortiert sind, sodass $i+1$ und $i-1$ immer definiert sind.}, unter Berücksichtigung der folgenden Regeln:
\begin{itemize}
\item[W1] Jede Kante ist entweder in eine oder zwei Richtungen orientiert. Falls sie bigerichtet ist haben beide Richtungen unterschiedliche Label.
\item[W2] An jeder Aufhängung  $a_i$ existiert eine nach aussen gerichtete Kante ohne Endpunkt mit Label i.  
\item[W3] Jeder Knoten $v$ hat hat Ausgangsgrad eins in jedem Label. Um $v$ existieren im Uhrzeigersinn eine Auskante mit Label 1, null oder mehr eingehende Kanten mit Label 3, eine Auskante mit Label 2, null oder mehr  eingehende Kanten mit Label 1, eine Auskante mit Label 2 und null oder mehr  eingehende Kanten mit Label 2.
\item[W4] Es existiert kein inneres Gebiet mit einem gerichteten Zykel in einer Farbe als Rand.
\end{itemize}
\end{definition}

\begin{figure}[h]
	\centering
  \includegraphics[width=0.8\textwidth]{schnyder_wood_def.png}
\end{figure}

Analog zu den Schnyder Woods, kann man Schnyder Labelings definieren, die zu diesen in Bijektion stehen. Hier betrachten wir nicht zuerst die Kanten eines planaren Graphen sondern die Winkel an den Knoten.

\begin{definition}[Schnyder Labeling]\label{def_sl}
Ein Schnyder Labeling ist eine Beschriftung der Winkel von $G$ mit den Labeln 1, 2 und 3 (oder rot, grün und blau) unter Berücksichtigung der folgenden Regeln:
\begin{itemize}
\item[L1] Um jedes innere Gebiet bilden die Label im Uhrzeigersinn nichtleere Intervalle von 1en, 2en und 3en. Am äusseren Gebiet gilt dies gegen den Uhrzeigersinn.
\item[L2] Um jeden inneren Knoten bilden die Label im Uhrzeigersinn nichtleere Intervalle von 1en, 2en und 3en.
\item[L3] An Aufhängung $a_i$ haben äusseren Winkel die Label i-1 und i+1 im Uhrzeigersinn mit der halben Auskante dazwischen und die inneren Winkel das Label i.
\end{itemize} 
\begin{figure}[h]
	\centering
  \includegraphics[width=0.8\textwidth]{schnyder_label_def.png}
\end{figure}
\end{definition}

In Abbildung \ref{schnyder_bij} wird eine Verbindung zwischen Schnyder Woods und Schnyder Labelings illustriert. Das nächste Lemma folgt aus L1 und L2.

\begin{lemma}\label{lem_sl}
Sei G ein planer, intern-3-zusammenhängender Graph mit den Aufhängungen $a_1,a_2,a_3$ und einem Schnyder Labeling. Dann beinhalten die vier Winkel entgegen dem Uhrzeigersinn an jeder Kante die Label 1, 2 und 3. Somit hat jede Kante einen der beiden Typen in Abbildung \ref{schnyder_bij}.
\end{lemma}

\begin{figure}[h]
	\centering
  \includegraphics[width=0.7\textwidth]{schnyder_bij.png}
	\caption{Bijektion zwischen Schnyder Wood auf der rechten und Schnyder Labeling auf der linken Seite.}
	\label{schnyder_bij}
\end{figure}

Wenn wir uns auf intern-3-zusammenhängende planare Graphen beschränken, dann ist die dargestellte Abbildung nach \cite[Theorem 2.3]{felsner04} eine Bijektion. Zu jedem Schnyder Labeling gehört also genau ein Schnyder Wood und anders herum. Dies macht es möglich, wenn sinnvoll, zwischen den beiden Strukturen hin und her zu wechseln. So kann es auch im Verlauf dieser Arbeit vorkommen, dass wir vom einen schreiben, aber implizit Eigenschaften des andern meinen.\\

Es existieren einige Anwendungen von Schnyder Woods im Bezug auf Einbettungen. Wie schon erwähnt bezieht sich eines der ersten Resultate auf die konvexe Einbettung auf einem Gitter. Eine Verbesserung des in \cite{schnyder90} erreichten ist das im Folgenden skizzierte \textit{face-counting} \cite{felsner01}. Betrachte $G$ mit einem Schnyder Wood $T_1,T_2,T_3$. Nach \cite[Korollar 2.5]{felsner04} handelt es sich bei den $T_i$ um gerichtete Bäume mit Wurzeln in $a_i$. Von jedem Knoten $v$ aus existierten also eindeutige Pfade $P_i(v)$ zu den Aufhängungen $a_i$. Die Pfade von $v$ zu den Aufhängungen treffen sich nach \cite[Lemma 2.4]{felsner04} nur in $v$. Wir erhalten also zu jedem Knoten $v$ drei Regionen $R_i$ , die jeweils von den Pfaden $P_{i-i}(v)$ und $P_{i+1}(v)$ und dem äusseren Gebiet eingegrenzt werden. In jeder dieser Regionen können wir nun die eingeschlossenen Gebiete von $G$ zählen. Durch das Zählen der Gebiete in den Regionen zu $v$ lässt sich eine konvexe Zeichnung von $G$ erzeugen.

Hierzu ordnet man jedem Knoten $v$ seien Gebiets Vektor $(v_1,v_2,v_3)$ zu, wobei $v_i$ die Anzahl der inneren Gebiete in $R_i(v)$ beschreibt. Nun gilt für jeden Knoten $v_1+v_2+v_3 = |F|-1$. Seien $\alpha_1 = (0,1),\alpha_2 = (1,0)$ und $\alpha_3 = (0,0)$ die äusseren Ecken unserer Zeichnung. Sie entsprechen ebenfalls den Bildern der Aufhängungen von $G$. Die Position der inneren Knoten ergibt sich nun durch die Funktion 
$$\mu: V \to \mathbb{R}^2,v\mapsto v_1\alpha_1 + v_2\alpha_2+v_3\alpha_3.$$ 

Nach \cite[Theorem 2.7]{felsner04} ist die mit diesen Koordinaten erzeugte Zeichnung planar und konvex und passt auf ein $(|F|-1)\times(|F|-1)$-Gitter. Sie hat noch eine weitere Eigenschaft die später von Nutzen sein wird und in Abbildung \ref{face_counting} dargestellt ist.

\begin{figure}
	\centering
  \includegraphics[width=0.8\textwidth]{face_counting.png}
	\caption{Eine Schnyder Wood und die durch \textit{face counting} erhaltene Einbettung. Die eingefärbten Gebiete sind die Regionen die den Gebietsvektor $(v_1,v_2,v_3)$ ergeben. In der Mitte ist W5 illustriert.}
	\label{face_counting}
\end{figure}

\begin{itemize}
\item [W5] Die Knoten eines inneren Gebietes werden auf die Seiten eines Dreiecks mit den Seiten $c_i(\alpha_{i-1}-\alpha_{i+1})$ mit passenden Konstanten $c_i$ abgebildet. Im inneren dieses Dreiecks befinden sich keine Knoten und die Winkel des Gebietes auf der Seite $c_i(\alpha_{i-1}-\alpha_{i+1})$ haben Label $i$ im Schnyder Labeling.
\end{itemize}



\section{$\alpha$-Orientierungen}\label{alpha_orientations}

Für unseren Algorithmus in Kapitel \ref{main_algo} führen wir eine weitere zu Schnyder-Woods und Labelings in Bijektion stehende Struktur auf Graphen ein und halten uns dabei an \cite{felsner04}.

Sei $G=(V,E)$ ein ungerichteter Graph und $\alpha:V\mapsto\mathbb{N}$ eine Funktion auf $G$. Eine $\alpha-Orientierung$ ist eine Orientierung der Kanten von $G$, sodass der Ausgrad eines jeden Knoten $\alpha(v)$ entspricht. Somit gilt $$outdeg(v) = \alpha(v).$$

Wir betrachten den Primal-Dual Graphen $G+G^*$ eines planen Graphen $G$. Hier ist $G^*$ der schwache duale Graph\footnote{$G^*$ hat einen Knoten für jedes innere Gebiet von $G$ und eine Kante verbindet zwei Knoten $f_1,f_2$ falls sie in $G$ adjazent sind.} zusammen mit einer Halbkante ins äussere Gebiet von jeder inzidenten Kante aus. Die Menge der Knoten von $G+G^*$ besteht aus Knoten-Knoten, Kanten-Knoten und Gebiets-Knoten. Kanten in $G+G^*$ existieren, sowohl zwischen inzidenten Kanten und Knoten, als auch Kanten und Gebieten in $G$. Somit ist $G+G^*$ bipartit. Falls wir einen Knoten $f_\infty$ für das äussere Gebiet einsetzten und die Halbkanten verlängern, spricht man vom Abschluss von $G+G^*$ und bezeichnet diesen mit $\tilde{G}$. Das folgende Theorem liefert eine Bijektion zwischen Schnyder Woods auf $G$ und einer bestimmten $\alpha$-Orientierung auf $\tilde{G}$, die wir $\alpha_s$ nennen.

\begin{theorem}\label{alpha_bij}
Sei $G$ ein planer Graph mit Aufhängungen $\{a_1,a_2,a_3\}$, dann sind die folgenden Strukturen in Bijektion:
\begin{itemize}
\item Die Schnyder Wälder auf $G$.
\item Die Schnyder Wälder auf dem (schwachen) dualen Graphen $G^*$.
\item Die $\alpha_{s}$-Orientierungen des Abschlusses von $G+G^*$ mit $\alpha_s(v) = \alpha_s(f) = 3$ für jeden Knoten- und Gebiets-Knoten,  $\alpha_s(e) = 1$ für jeden Kanten-Knoten und  $\alpha_s(f_\infty) = 0$.
\end{itemize}
\end{theorem}

\begin{figure}[h]
	\centering
  \includegraphics[width=0.95\textwidth]{alpha_ex.png}
  \caption{Der Primal-Duale Graph $K_4+K_4^*$ mit einer $\alpha_s$-Orientiertung und dem zugehörigen Schnyder Wood auf $K_4$. }
\end{figure}
\section{Flüsse auf Graphen}

Wir werden in Kapitel \ref{main_algo} einen gerichteten Graphen $\mathcal{N}$ konstruieren um auf diesem einen maximalen Fluss $\varphi$ zu finden. Es gibt sehr viele unterschiedliche Arten von Flussproblemen. So kann man zum Beispiel Graphen mit nur einem Paar von Quellen und Senken oder mehreren betrachten und die Kanten können gerichtet oder unterrichtet sein. Im Fall mit mehreren Quellen und Senken werden diese normalerweise als Paare $s_i,t_i$ gehandhabt und es wird gefordert, dass insgesamt Fluss $\varphi_i$ mit Stärke $d_i \in \mathbb{R}_+$ von $s_i$ zu $t_i$ fließt. Als zusätzliche Einschränkung haben die Kanten $e$ maximale Kapazitäten $c(e) \in \mathbb{R}_+$ die nicht überschritten werden können. Für jede Kante muss also gelten $\varphi(e) \leq c(e)$. Wir werden uns in Kapitel \ref{main_algo} mit einem Fluss der folgenden Form befassen.

\begin{definition}[Gerichtetes-Multi-Fluss-Problem]\label{def_multi_flow}
Sei $D=(V,E)$ ein gerichteter Graph, im Weiteren auch Netzwerk genannt, mit den Kapaziäten $c:E\mapsto\mathbb{R}_{+}$, Paaren von ausgezeichneten Knoten $\{(s_1,t_1), ... ,(s_n,t_n)\}$ und positiven Bedarfen $\{d_1, ... ,d_n\}$, dann ist $\varphi=(\varphi_1, ... ,\varphi_n)$ ein zulässiger Fluss, falls
\begin{itemize}
\item[F1] $\forall (u,v) \in E : \sum_{i=1}^{n}{\varphi_i(u,v)} \leq c(u,v) $
\item[F2] $ \forall u \neq s_i,t_i : \sum_{w \in V} \varphi_i(u,w) - \sum_{w \in V} \varphi_i(w,u) $
\item[F3] $ \forall s_i : \sum_{w \in V} \varphi_i(s_i,w) - \sum_{w \in V} \varphi_i(w,s_i) = d_i $
\item[F4] $ \forall t_i : \sum_{w \in V} \varphi_i(w,s_i) - \sum_{w \in V} \varphi_i(s_i,w) = d_i $
\end{itemize}
\end{definition}

Es folgen zwei bekannte Resultate für den Fall $n=1$, die später Anwendung finden werden.

\begin{theorem}[Max-Flow Min-Cut]
$\varphi$ ist ein maximale Fluss auf $\mathcal{N}$, genau dann, wenn für mindestens einen Schnitt $\mathcal{S} \subset E$ gilt $c(\mathcal{S}) = |\varphi|$. Die Kapazität eines minimalen Schnittes entspricht dem maximalen Fluss.
\end{theorem}

\begin{theorem}[Ganzzahliger Fluss]\label{theo_int_flow}
Sei $\mathcal{N}$ ein Netzwerk mit einer Quelle und einer Senke und alle Kapazitäten seien ganzzahlig, dann existiert auch ein maximaler Fluss $\varphi$, sodass der Fluss auf allen Kanten ganzzahlig ist. Es gilt also $|\varphi(e)| \in \mathbb{N}$ für alle $e\in E$.
\end{theorem}

\begin{remark}
Im Fall $n=1$ und Kapazitäten $c:E\mapsto\mathbb{N}$ impliziert die Existenz eines zulässigen Flusses die Existenz einer ganzzahligen Lösung, sowohl für gerichtete als auch ungerichtete Graphen. Diese lässt sich in polynomineller Zeit bestimmen. Für $n=2$ und ungerichtete Graphen gilt dies nach \cite{hu} ebenfalls. Für diese Arbeit wäre jedoch der Fall $n=2$ für gerichtete Graphen interessant. Leider ist hier im Allgemeinen die Suche nach einer ganzzahligen Lösung nur über Gemischte Lineare Programmierung (TODO) möglich und befindet sich somit in $\mathcal{NP}$. Es existieren ebenfalls keine analogen Aussagen zum Max-Flow Min-Cut Theorem für gerichtete Netzwerke mit mehr als einer Quelle und Senke, sondern nur Schranken und Annäherungen \cite{leighton99}.
\end{remark}
