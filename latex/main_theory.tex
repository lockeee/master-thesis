\chapter{Hauptteil}

In diesem Kapitel wollen wir uns mit notwendigen und hinreichenden Bedingungen für die Existenz von SLTRs auseinandersetzten. Hierzu führen wir zuerst eine in gewisser Weise schwächere Bedingung für planare Graphen ein.

\begin{definition}[FAA]
Sei $G=(V,E,F)$ ein planer Graph, dann ist eine Flache Winkel Zuordnung, im weiteren (nach dem englischen \textit{flat angle assignment}) mit FAA bezeichnet, ein Matching zwischen Knoten und Gebieten, sodass:
\begin{itemize}
\item [A1] Jedem Gebiet $f$ sind genau $deg(f)-3$ Knoten zugeordnet.
\item [A2] Jeder Knoten $v$ ist höchstens einem Gebiet zugeordnet.
\end{itemize}
Für den Fall das wir einen Graph mit Aufhängungen betrachten, dann fordern wir zusätzlich:
\begin{itemize}
\item [A3] Die inzidenten Knoten am äusseren Gebiet, die keine Aufhängungen sind, müssen dem äusseren Gebiet zugeordnet werden.
\end{itemize}
\end{definition}

Somit gibt uns ein SLTR auch ein FAA, in die andere Richtung gilt dies aber im allgemeinen nicht.

%% Bild

\section{Harmonische Funktionen}


\section{Ecken kompatible Paare}