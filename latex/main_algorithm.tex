\chapter{Algorithmen zur Erkennung von SLTRs}\label{main_algo}

Im vorherigen Kapitel wurden Kriterien für die Existenz einer SLTR erarbeitet, die allerdings nicht sofort einen Algorithmus, sowohl zur Frage nach der Existenz, als auch zum Erlangen einer spezifischen SLTR liefern. Diesem Thema wollen wir uns nun in diesem Kapitel zuwenden und dafür zum Einstieg einen von Aerts und Felsner in \cite{af15} erarbeiteten Algorithmus betrachten.

\section{SLTRs via Zwei-Fluss}

Wir betrachten im folgenden gerichtete Graphen. Das Ziel ist es, für einen gegebenen Graphen sowohl einen Schnyder Wood als auch ein FAA jeweils als Lösung eines Fluss-Problems zu erhalten und diese beiden dann in einem Zwei-Fluss-Problem zu kombinieren, sodass eine Lösung ein Ecken-Kompatibles-Paar gibt und wir somit eine SLTR erhalten. Wir beschäftigen uns also mit der folgenden Problemstellung.

\begin{definition}[Gerichtetes-Multi-Fluss-Problem]
Sei $D=(V,E)$ ein gerichteter Graph, im Weiteren auch Netzwerk genannt, mit den Kapaziäten $c:E\mapsto\mathbb{R}_{+}$, Paaren von ausgezeichneten Knoten $\{(s_1,t_1), ... ,(s_n,t_n)\}$ und positiven Bedarfen $\{d_1, ... ,d_n\}$, dann ist $\varphi=(\varphi_1, ... ,\varphi_n)$ ein zulässiger Fluss, falls
\begin{itemize}
\item[F1] $\forall (u,v) \in E : \sum_{i=1}^{n}{\varphi_i(u,v)} \leq c(u,v) $
\item[F2] $ \forall u \neq s_i,t_i : \sum_{w \in V} \varphi_i(u,w) - \sum_{w \in V} \varphi_i(w,u) $
\item[F3] $ \forall s_i : \sum_{w \in V} \varphi_i(s_i,w) - \sum_{w \in V} \varphi_i(w,s_i) = d_i $
\item[F4] $ \forall t_i : \sum_{w \in V} \varphi_i(w,s_i) - \sum_{w \in V} \varphi_i(s_i,w) = d_i $
\end{itemize}
\end{definition}

\begin{remark}
Im Fall $n=1$ und Kapazitäten $c:E\mapsto\mathbb{N}$ impliziert die Existenz eines zulässigen Flusses die Existenz einer ganzzahligen Lösung, sowohl für gerichtete als auch ungerichtete Graphen, und diese lässt sich in polynomineller Zeit bestimmen. Für $n=2$ und ungerichtete Graphen gilt dies nach \cite{hu} ebenfalls. Für uns im Folgenden interessant wäre jedoch, wie wir sehen werden, der Fall $n=2$ für gerichtete Graphen. Leider ist hier im Allgemeinen die Lösung nur über Lineare Programmierung möglich und befindet sich somit in $\mathcal{NP}$.
\end{remark}

\subsection{Schnyder-Wood-Fluss}

Um einen Schnyder Wood als Fluss-Problem zu kodieren, kann man die in Abschnitt \ref{alpha_orientations} eingeführten $\alpha_s$-Orientierungen auf dem Abschluss von $G+G^*$ nutzen. Fusy zeigt in \cite{fusy07} im Zuge der Untersuchung spezifischer $\alpha$-Funktionen, dass sich $\alpha_s$-Orientierungen von $G+G^*$ in linearer Zeit berechnen lassen, sodass wir auch einen Schnyder Wood auf $G$ in linearer Zeit erhalten.

Sei ein ebener intern-3-zusammenhängender Graph $G$ mit Aufhängungen $a_1,a_2,a_3$ gegeben. Machen wir uns an die Konstruktion eines Netzwerks $\mathcal{N}_S$, sodass ein zulässiger ganzzahliger Fluss $\varphi_s$ auf $\mathcal{N}_S$ einer $\alpha_s$-Orientierung von $G+G^*$ entspricht\footnote{Hier stehen die drei \textit{s} jeweils für Schnyder um spätere Verwechslungen zu vermeiden.}. Nach Theorem \ref{alpha_bij} würde die aus $\varphi_s$ erhaltene $\alpha_s$-Orientierung somit auch einen Schnyder Wood auf $G$ induzieren. Besonderes Augenmerk ist hier auf die Möglichkeit einer späteren Kombination mit einem weiteren Netzwerk gelegt, in welchem eine ganzzahlige Lösung einem FAA auf $G$ entspricht, um ein kombiniertes Netzwerk zu erstellen.

Wie schon erwähnt ist $G+G^*$ bipartit, Kanten-Knoten haben Grad 4, Knoten-Knoten haben Grad $\text{deg}(v)$ und Gebiets-Knoten haben Grad $|f|$. Für eine $\alpha_s$-Orientierung muss nach Definition \ref{alpha_bij} A3 jeder Kanten-Knoten Ausgangsgrad 1, jeder Knoten-Knoten Eingangsgrad $\text{deg}(v)-3$ und jeder Gebiets-Knoten Eingangsgrad $|f|-3$ haben. Die Kanten-Knoten am äußeren Gebiet sind in $G+G^*$ immer nach außen orientiert. Somit reicht es, wenn wir nur die inneren Kanten-Kanten $E_{in}$ betrachten und die Orientierung der äußeren Kanten voraussetzten.

\begin{figure}[h]
	\centering
  \includegraphics[width=0.9\textwidth]{schnyder_flow.png}
  \caption{Der Schnyder Wood Fluss durch eine innere Kante $(v,w)$. Die nicht beschrifteten Kanten haben Kapazität 1.}
  \label{schnyder_flow}
\end{figure}

Das im Folgenden konstruierte Netzwerk ist in Abbildung \ref{schnyder_flow} skizziert und unter dem Punkt Netzwerk \ref{net_schnyder} zusammengefasst. Das Netzwerk $\mathcal{N}_S$ hat eine Quelle $s$ und eine Senke $t$ und einen Knoten für jeden Knoten aus $G+G^*$ (bis auf die äußeren Kanten). Es existieren Kanten von der Quelle zu jedem Kanten-Knoten $e \in E_{in}$ mit Kapazität 1. Von den Kanten-Knoten $e$ zu inzidenten Knoten-Knoten $v$ und Gebiets-Knoten $f \in F_{in}$ fügen wir ebenfalls Kanten mit Kapazität 1 ein. Merke, dass der Fluss auf diesen Kanten bei einer ganzzahligen Lösung eine $\alpha_s$-Orientierung auf $G+G^*$ ergeben soll. Nun fügen wir noch Kanten von $f \in F_{in}$ zur Senke mit Kapazitäten $|f|-3$, Kanten von den (inneren) Knoten-Knoten $v \in V_{in} = V \setminus \{a_1,a_2,a_3\}$ zur Senke mit Kapazitäten $\text{deg}(v)-3$ und Kanten von den Aufhängungen $a_i$ zur Senke mit Kapazitäten $\text{deg}(a_i)-2$. Die letzte Kapazität ergibt sich, da die Halbkante in $G+G^*$ von $a_i$ aus immer nach außen orientiert. Wir müssen somit nur noch zwei andere Kanten von $a_i$ weg orientieren. Der Bedarf des Netzwerkes entspricht der Anzahl der inneren Kanten von $G$. Fassen wir zusammen.

\begin{network}[Schnyder Wood]\label{net_schnyder}
Bei $\mathcal{N}_S$ handelt es sich um ein gerichtetes Netzwerk, das auf Basis eines ebenen intern-3-zusammenhängenden Graphen $G$ mit Aufhängungen $a_1,a_2,a_3$ erstellt wird, um einen Schnyder Wood auf $G$ zu finden. Ein Ausschnitt ist in Abbildung \ref{schnyder_flow} dargestellt.
	\begin{itemize}
	\item $\mathcal{N}_S$ hat eine Quelle $s$ und eine Senke $t$
	\item Knoten in $\mathcal{N}_S$ werden für jeden innere Kante $e \in E_{in}$, jedes innere Gebiet $f\in F_{in}$ und jeden Knoten $v \in V$ aus $G$ erzeugt.
	\item Es werden gerichtete Kanten der folgenden Typen in $\mathcal{N}_S$ erzeugt:
		\begin{itemize}
		\item $(s,e)$ von der Quelle zu jeder inneren Kante mit $c\big(s,e\big) = 1$
		\item $(e,v_1),(e,v_2)$ von jeder inneren Kante zu den Endknoten mit $c\big(e,v\big) = 1$
		\item $(e,f)$ von jeder inneren Kante zu adjazenten Gebieten mit $c\big(e,f\big) = 1$
		\item $(v,t)$ von den Knoten zur Senke mit $c\big(v,t\big) = 1$
		\item $(f,t)$ von den inneren Gebieten zur Senke mit $c\big(f,t\big) = |f|-3$
		\item $(a_i,t)$ von den Aufhängungen zur Senke mit $c\big(f,t\big) = \text{deg}(a_i)-2$
		\item $(v,t)$ von den restlichen Knoten zur Senke mit $c\big(f,t\big) = \text{deg}(v)-3$
		\end{itemize}
	\item $\mathcal{N}_S$ hat Bedarf $d=|E_{in}|$
	\item [$\Rightarrow$] Ein zulässiger ganzzahliger Fluss $\varphi_s$ existiert genau dann, wenn ein Schnyder Wood  auf $G$ existiert.
	\end{itemize}
\end{network}

\begin{proposition}
Sei $G$ ein ebener intern-3-zusammenhängender Graph mit Aufhängungen $a_1,a_2,a_3$ und $\mathcal{N}_S$ wie in Netzwerk \ref{net_schnyder} konstruiert. Ein ein zulässiger ganzzahliger Fluss $\varphi_s$ auf $\mathcal{N}_S$ existiert genau dann, wenn ein Schnyder Wood auf $G$ zu den Aufhängungen $a_1,a_2,a_3$ existiert. Insbesondere kodiert jeder Fluss einen Schnyder Wood.
\end{proposition}

\begin{proof}
Sei $\varphi_s$ ein zulässiger ganzzahliger Fluss auf $\mathcal{N}_S$. Es gilt $|\varphi_s| = |E_{in}|$. Somit hat jede innere Kante-Knoten Ausgangsgrad 1. Weiter gilt 
$$\sum_{v \in V} \Big(\text{deg}(v)-3\Big) + 3 + \sum_{f \in F_{in}} \Big(|f|-3\Big) = |E_{in}|.$$
Somit sind die Kanten von den Kanten-Knoten und Knoten-Knoten ausgelastet. Der Fluss $\varphi_s$ entlang einer Auskante von $e \in E_{in}$ in $\mathcal{N}_S$ entspricht dann genau der Kante in $G+G^*$, die in der $\alpha_{s}$-Orientierung von $e$ weg orientiert wird. Durch die Knoten-Knoten $v$ und Gebiets-Knoten $f$ fließt Fluss auf $\text{deg}(v)-3$ bzw. $|f|-3$ Kanten der von den Kanten-Knoten kommt. Somit lässt $\varphi_s$ jeweils drei Kanten frei, denen wir noch keine Orientierung zugewiesen haben. Diese entsprechen somit den von $v$ bzw. $f$ weg orientierten Kanten einer $\alpha_{s}$-Orientierung auf $G+G^*$. Somit hat jedes Gebiet und jeder Knoten den passenden Ausgangsgrad und $\varphi_s$ kodiert eine $\alpha_s$-Orientierung auf $G+G^*$. Auf analoge Weise lässt sich aus einer $\alpha_s$-Orientierung ein zulässiger ganzzahliger Fluss $\varphi_s$ erstellen. Es existiert also genau dann ein Schnyder Wood auf $G$, wenn eine ganzzahlige Lösung $\varphi_s$ für $\mathcal{N}_S$ existiert.
\end{proof}

\subsection{FAA-Fluss}\label{faa-flow}

Sei wieder ein ebener intern-3-zusammenhängender Graph $G$ mit Aufhängungen $a_1,a_2,a_3$ gegeben. Ein FAA ordnet nun jedem Gebiet $|f|-3$ flache Winkel zu und jeder Knoten darf maximal einem Gebiet zugeordnet werden. Die adjazenten Knoten um das äußere Gebiet, die keine Aufhängungen sind, müssen diesem zugewiesen werden. Wir konstruieren ein Netzwerk $\mathcal{N}_F$, für das ein ganzzahliger zulässiger Fluss einem FAA entspricht 

\begin{figure}
	\centering
  \includegraphics[width=0.8\textwidth]{faa_flow.png}
  \caption{Der FAA-Fluss durch einen Winkel $(f,v)$. Die nicht beschrifteten Kanten haben Kapazität 1.}
  \label{faa_flow}
\end{figure}

Das im Folgenden konstruierte Netzwerk ist in Abbildung \ref{faa_flow} skizziert und unter dem Punkt Netzwerk \ref{net_faa} zusammengefasst. Das Netzwerk $\mathcal{N}_F$ hat jeweils eine Quelle und Senke. Wir erstellen in $\mathcal{N}_F$ Knoten für jeden inneren Winkel $(f,v)$, mit $v\in V$ und $f \in F_{in}$, Knoten für alle inneren Gebiete $f \in F_{in}$ und für die inneren Knoten $v\in V_{in}$. Von der Quelle fügen wir Kanten mit Kapazität 1 zu jedem inneren Winkel $(f,v)$, und von jedem inneren Winkel $(f,v)$ jeweils eine Kante zu $f$ und zu $v$ (falls $v\in V_{in}$) mit Kapazität 1 ein. Merke, dass so der Fluss von den Winkeln zu den inneren Knoten einem FAA entspricht. Zuletzt erstellen wir Kanten von jedem inneren Gebiet $f$ zur Senke mit Kapazität 3 und Kanten von jedem Knoten $v$ zur Senke mit Kapazität 1 ein. Der Bedarf des Netzwerks ist $\sum_{f \in F_{in}}|f|$ und entspricht der Anzahl der inneren Winkel von $G$. Fassen wir zusammen.

\begin{network}[FAA]\label{net_faa}
Bei $\mathcal{N}_F$ handelt es sich um ein gerichtetes Netzwerk, das auf Basis eines ebenen intern-3-zusammenhängenden Graphen $G$ mit Aufhängungen $a_1,a_2,a_3$ erstellt wird, um ein FAA zu finden. Ein Ausschnitt ist in Abbildung \ref{faa_flow} dargestellt.
	\begin{itemize}
	\item $\mathcal{N}_F$ hat eine Quelle $s$ und eine Senke $t$
	\item Knoten in $\mathcal{N}_F$ werden für jeden inneren Winkel $(f,v) \in W_{in}$, jedes innere Gebiet $f\in F_{in}$ und jeden Knoten $v \in V$ aus $G$ erzeugt.
	\item Es werden gerichtete Kanten der folgenden Typen in $\mathcal{N}_F$ erzeugt:
		\begin{itemize}
		\item $(s,(f,v))$ von der Quelle zu jedem inneren Winkel mit $c\big(s,(f,v)\big) = 1$
		\item $((f,v),v)$ von jedem inneren Winkel zum Knoten mit $c\big((f,v),v\big) = 1$
		\item $((f,v),f)$ von jedem inneren Winkel zum Gebiet mit $c\big((f,v),f\big) = 1$
		\item $(f,t)$ von den inneren Gebieten zur Senke mit $c\big(f,t\big) = 3$
		\item $(v,t)$ von den Knoten zur Senke mit $c\big(f,t\big) = 1$
		\end{itemize}
	\item $\mathcal{N}_F$ hat Bedarf $d=\sum_{f \in F_{in}}|f|$
	\item [$\Rightarrow$]Ein zulässiger ganzzahliger Fluss $\varphi_F$ existiert genau dann, wenn ein FAA existiert.
	\end{itemize}
\end{network}

\begin{proposition}
Sei $G$ ein ebener intern-3-zusammenhängender Graph mit Aufhängungen $a_1,a_2,a_3$ und $\mathcal{N}_F$ wie in Netzwerk \ref{net_faa} konstruiert. Ein ein zulässiger ganzzahliger Fluss $\varphi_F$ auf $\mathcal{N}_S$ existiert genau dann, wenn ein FAA auf $G$ zu den Aufhängungen $a_1,a_2,a_3$ existiert. Insbesondere kodiert jeder Fluss genau ein FAA.
\end{proposition}

\begin{proof}
Sei $\varphi_F$ ein zulässiger ganzzahliger Fluss auf $\mathcal{N}_F$. Der Fluss auf einer Kante $((f,v),f)$ entspricht einer Ecke von $f$ und Fluss auf $((f,v),u)$ der Zuweisung eines Knoten zu $f$. Zur Vereinfachung sprechen wir im Weitern auch von Ecken- und Zuweisungs-Fluss. Somit wird jeder innere Winkel von $\varphi_F$ entweder einem Gebiet zugewiesen oder als Ecke ausgezeichnet. Da die Kanten von den inneren Knoten zur Senke Kapazität eins haben kann jeder Knoten nur einmal zugewiesen werden. F2 ist somit erfüllt. Von jedem inneren Gebiet $f$ fließt Fluss mir Stärke $|\varphi_s(f,t)| = 3$ zu Senke. Somit muss $|f|-3$ Fluss durch zugewiesene Knoten gehen. F1 ist somit erfüllt. $\varphi_F$ respektiert also die Bedingungen aus Definition \ref{def_faa} und es existieren nur dann FAAs auf $G$, wenn mindestens eine ganzzahlige Lösung für $\mathcal{N}_F$ existiert.
\end{proof}

\begin{remark}

Das oben eingeführte Netzwerk \ref{net_faa} zur Bestimmung von FAAs lässt sich auch als Zwei-Fluss-Problem konstruieren. Wir trennen für Ecken- und Zuweisungs-Fluss die Quellen und Senken.  Zu jedem Winkelknoten $(f,v)$ fügen wir eine Kopie $(f,v)'$ ein und verbinden beide durch eine Kante mit Kapazität 1 (siehe Abbildung \ref{faa_as_2}). In den ersten führt eine Kante von beiden Quellen mit Kapazität 1. Und vom zweiten aus führen Kanten zu $f$ und $v$. Der Bedarf des Ecken-Flusses ist dann $3|F_{in}|$ und der Bedarf des Zuweisung-Flusses $\sum_{f \in F_{in}}{|f|-3}$.

\begin{figure}[h]
	\centering
  \includegraphics[width=0.8\textwidth]{faa_2_flow.pdf}
  \caption{Das einfügen einer Kante und eines zweiten Winkel Knotens gewährt die Trennung von Winkel- und Zuweisungs-Fluss. Die nicht beschrifteten Kanten haben Kapazität 1.}
  \label{faa_as_2}
\end{figure}

Eine zulässige ganzzahlige Lösung $\varphi_F = (\varphi_e,\varphi_z)$ entspricht dann wieder einem FAA auf $G$. Aus der Ganzzahligkeit folgt, dass ein Winkel entweder von $\varphi_{e}$ (Ecke) oder $\varphi_{z}$ (Zuweisung) genutzt wird. Dies gewährleisten die Kanten zwischen den Winkelknoten, da sie immer nur von einem der beiden Flüssen genutzt werden können.
\end{remark}




\subsection{Graphen mit wenigen FAAs oder Schnyder Woods}

Wir wollen in diesem Abschnitt kurz betrachten, wie wir 
Sei $G$ ein intern-3-zusammenhängender Graph 
Für Graphen mit wenigen FAAs oder wenigen Schnyder Woods\footnote{Es sind hier maximal polynominell viele gemeint.} liefern die Ansätze oben jeweils Wege in polynomineller Zeit zu verifizieren, ob SLTRs existieren und ein Gutes-FAA zu erhalten. Die Graphen mit nur genau einen Schnyder Wood wurden von Felsner und ... TODO 

\subsection{Ein Zwei-Fluss Netzwerk zur Erkennung von SLTRs}

Im Verlauf des Kapitels haben wir nun sowohl für Schnyder Woods als auch für FAAs ein Netzwerk betrachtet, für das eine ganzzahlige Lösung einen Schnyder Wood bzw. ein FAA für einen planen Graphen $G$ liefert. Wir wollen jetzt eine Kombination aus beiden erstellen die ein Ecken kompatibles Paar $(\sigma,\phi)$ aus einem Schnyder Labeling $\sigma$ und einem FAA $\phi$  kodiert.\

Es folgt die Konstruktion eines Netzwerkes, wir bezeichnen es mit $\mathcal{N}_G$, welches diesen Wunsch erfüllt, für das eine ganzzahlige Lösung ein Ecken kompatibles Paar kodiert und somit nach Theorem \ref{theo_coco} eine SLTR für $G$ existiert. Leider handelt es sich hierbei um ein 2-Fluss-Netzwerk, aber darauf wollen wir später genauer eingehen.\\

Wie oben in Abschnitt \ref{faa-flow} erwähnt lässt sich ein FAA auch mit einem Zwei-Fluss kodieren und wir können Ecken- und Zuweisungs-Fluss mit den passenden Bedarfen getrennt betrachten. Wir müssen jetzt diese drei Flüsse, also Schnyder-, Ecken- und Zuweisungs-Fluss in einem Netzwerk kombinieren. In \cite{af15} ergeben Schnyder- und Ecken-Fluss zusammen Fluss von Typ 1 und der Zuweisungs-Fluss Typ 2. Wir wollen hier analog ein Netzwerk konstruieren in dem wir FAA und Schnyder-Wood Fluss nicht trennen. Der Verständlichkeit halber werden wir Pfade, die in einer Lösung von einem der drei Flussarten genutzt werden, \textit{Schnyder-Pfad, Ecken-Pfad} und \textit{Zuweisungs-Pfad} nennen.\\

Bei der Kombination der beiden oben konstruierten Netzwerke $\mathcal{N}_S$ und $\mathcal{N}_F$ zu $\mathcal{N}_G$ müssen wir die Ecken Kompatibilität gewährleisten. K1 zu erfüllen, also die Nutzung der gleichen Aufhängungen von $\sigma$ und $\phi$, ist kein Problem. Allerdings müssen wir für K2 das Netzwerk etwas komplizierter machen. Betrachten wir als Basis $\mathcal{N}_S \cup \mathcal{N}_F$ und fürs erste nur ein inneres Gebiet $f$, dann sehen wir, dass es $|f|-3$ Schnyder-Fluss aufnimmt, aber $|f|$ Einkanten in $\mathcal{N}_S$ hat, es sind also genau die drei nötigen Kanten für den Ecken-Fluss aus $\mathcal{N}_F$ übrig. Wir müssen gewährleisten, dass jede Ecke im Schnyder Labeling ein anderes Label hat. Betrachten wir also den von $\varphi_S$ induzierten Schnyder Wood auf $G^*$. Nach \cite{felsner12} können wir diesen aus der $\alpha_S$-Orientierung ablesen. 

\begin{figure}[h]
	\centering
  	\includegraphics[width=0.7\textwidth]{alpha_bij.png}
  	\caption{a) Eine $\alpha_s$-Orientierung um eine innere Kante von $G$. b) Teile der korrespondierenden Schnyder Woods auf $G$ und $G^*$. c) Die induzierten Label, die für $G$ und $G^*$ gleich sind.}
	\label{alpha_bij}
\end{figure}

Es gilt außerdem, wie in Abbildung \ref{alpha_bij} skizziert, dass die Label der Ecke eines Gebietes in $G$ und das ihr in $G+G^*$ gegenüberliegenden Label der Ecke eines Gebiets um einen Knoten in $G^*$ gleich sind. Für eine zu $v$ in $G^*$ hin orientierte Kante folgt aus der Bijektion zwischen Schnyder Labelings und Schnyder Woods aus Abschnitt \ref{sw}, dass die Label links und rechts am Ende dieser Kante gleich sind. Somit sind auch die Label in $G$ gleich und wir können die folgende Eigenschaft festhalten.
\begin{itemize}
\item [A1] Die Label, des von $\alpha_s$ induzierten Schnyder Wood auf $G$, sind zwischen zwei aufeinander folgenden zu $f$ orientierten Kanten sind gleich.
\end{itemize}
Da es genau drei zu $f$ orientierte Kanten gibt müssen wir also dafür sorgen, dass für jedes Paar dieser Kanten eine Ecke zwischen ihnen liegt. Um dies zu erlangen implementieren wir eine zyklische Struktur um jedes innere Gebiet, wie in Abbildung \ref{comb_face_sketch} skizziert.\\

\begin{figure}[h]
	\centering
  	\includegraphics[width=0.9\textwidth]{combined_face_sketch.png}
  	\caption{Eine Skizze des kombinierten Netzwerkes auf einem inneren Gebiet mit $|f| = 4$. Beispielhaft sind Schnyder-Fluss (rot), Ecken-Fluss (blau) und Zuweisungs-Fluss (grün) eingezeichnet. }
	\label{comb_face_sketch}
\end{figure}

Betrachten wir zuerst den Schnyder-Fluss. Dieser wird Fluss von Typ 1, also von Quelle 1 zu Senke eins sein. Für einen Schnyder-Pfad der $\mathcal{N}_G$ durch einen Knoten $v$ verlässt hat sich nichts geändert. Der in der Skizze eingezeichnete Schnyder-Pfad der durch $f$ führt passiert davor einen extra Knoten, wir nennen ihn \textit{kleines Quadrat} der gewährleisten soll, dass von jeder Kante des Gebietes aus entweder ein Schnyder-Pfad oder ein Ecken-Pfad in $f$ mündet. Somit kodiert hier eine ganzzahlige Lösung weiterhin einen Schnyder Wood auf $G$.\

Kommen wir nun zum FAA-Fluss, also Fluss von Typ 2. Von Quelle 2 geht genau wie in Abbildung \ref{faa_flow} eine Kante zu jedem inneren Winkel $(f,v)$. Ein Zuweisungs-Pfad verlässt diesen Winkel über einen zusätzlich zu $v$ eingefügten Dummy-Knoten $v^*$. Von jedem $v^*$ geht eine Kante mit Kapazität 1 zu einer Dummy-Senke und von dieser eine Kante mit Kapazität $\sum_{f \in F_{in}} |f|-3$ zu Senke 2, wie in Abbildung \ref{dummy_sink} illustriert.

Die Dummy-Knoten sorgen dafür, dass jeder Knoten im FAA nur einmal zugewiesen werden kann, ohne in Konflikt mit dem Schnyder Wood-Fluss zu kommen, und die eingeschobene Dummy-Senke beschränkt die Anzahl der zugewiesenen Knoten, genau wie im zuvor konstruierten FAA-Fluss, auf $\sum_{f \in F_{in}} |f|-3$.
\begin{figure}[h]
	\centering
  	\includegraphics[width=0.7\textwidth]{dummy_sink.png}
  	\caption{Der Zuweisungsfluss durch die Winkel, Dummy-Knoten und die zusätzliche Kante vor Senke 2. Die Kante rechts hat Kapazität $\sum_{f \in F_{in}} |f|-3$ und alle anderen Kapazität 1.}
	\label{dummy_sink}
\end{figure}
Es bleibt der Ecken-Fluss. Hier betritt der Pfad wieder durch einen Winkel das Gebiet $f$ und muss es über ein ungenutztes kleines Quadrat verlassen. Die zweite und dritte Kante in jedem Winkeldreck gewährleisten, dass nicht immer das nächste kleine Quadrat genutzt werden muss. Falls dies von Schnyder-Fluss besetzt ist und der nächste Winkel zugewiesen wird, dann kann der Ecken-Pfad den nächsten Winkel passieren. Weiter sorgt die erste Kante, die von sowohl Schnyder-, als auch Winkel-Pfaden genutzt werden kann für eine eindeutige Beschriftung im Falle einer ganzzahligen Lösung.\\

Betrachten wir die Bedarfe der beiden Flüsse von Typ 1 und Typ 2, $\varphi_1$ bzw. $\varphi_2$. Beide entsprechen jeweils den Bedarfen der oben konstruierten $\mathcal{N}_S$ und $\mathcal{N}_F$, da nur so, mit den gleichen Argumenten wie zuvor, ein Schnyder Wood und ein FAA kodiert werden können. Jedes Gebiet benötigt genau drei Ecken und $|f|-3$ zugewiesene Knoten und der je ein Schnyder-Pfad führt durch jede innere Kante, $|E_{in}|$. Hier seien wieder $E_{in}$ die inneren Kanten und $F_{in}$ die inneren Gebiete von $G$. Es gilt also:

\begin{itemize}
\item Bedarf$(\varphi_1) = $ Bedarf$(\varphi_S) = |E_{in}|$
\item Bedarf$(\varphi_2) = $ Bedarf$(\varphi_F) =  \sum_{f \in F_{in}}(|f|-3) + 3|F_{in}| = \sum_{f \in F_{in}} |f|$
\end{itemize}

Bevor wir in Theorem \ref{theo_algo} zeigen, dass eine ganzzahlige Lösung $\varphi=(\varphi_1,\varphi_2)$ auch wirklich ein Ecken kompatibles Paar kodiert wollen wir noch ein Paar weitere Beobachtungen festhalten. Nehmen wir also an wir haben eine ganzzahlige Lösung $\varphi$ gefunden, dann gilt für diese:
\begin{itemize}
\item [A2] Jede äussere Kante in einem Winkel-Dreieck ist ausgelastet, sie wird entweder von einem Ecken- oder Zuweisungspfad genutzt.
\item [A3] Jede Kante von einem kleinen Quadrat zu einem inneren Gebiet $f$ ist ausgelastet, sie wird entweder von einem Schnyder- oder Ecken-Pfad genutzt.
\item [A4] Ein inneres Gebiet $f$ mit $|f|=3$ kann nicht von Zuweisungs- bzw. Schnyder-Pfaden genutzt werden.
\end{itemize}

Wir wollen diese Beobachtungen kurz begründen. Für jede mögliche ganzzahlige Lösung $\varphi$ gilt $$|\varphi|=|\varphi_1|+|\varphi_2| = |E_{in}| + \sum_{f \in F_{in}} |f|.$$
Da es genau $\sum_{f \in F_{in}} |f|$ innere Winkel gibt und der FAA-Fluss $\mathcal{N}_G$ nur durch diese betreten kann ergibt sich A2. Weiterhin führen $|f|-3$ Schnyder-Pfade durch jedes innere Gebiet, die $|f|-3$ der kleinen Quadrate nutzen, und die übrig gebliebenen werden von Ecken-Pfaden gebraucht, somit folgt A3. Durch ein inneres Gebiet $f$ müssen drei Ecken-Pfade führen und im Fall $|f|=3$ führt dies zu A4, da kein Platz in den Winkeln für Zuweisungs-Pfade und keine freien kleinen Quadrate für Schnyder-Pfade existieren.

\begin{theorem}
Sei $G$ ein intern-3-zusammenhängender Graph mit gegebenen Aufhängungen $\{a_1,a_2,a_3\}$, dann existiert eine SLTR von $G$, genau dann wenn ein ganzzahliger zulässiger Fluss $\varphi=(\varphi_1,\varphi_2)$ auf $\mathcal{N}_G$ existiert.
\end{theorem}

