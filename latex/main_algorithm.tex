\chapter{Algorithmen zur Erkennung von SLTRs}\label{main_algo}

Im vorherigen Kapitel wurden Kriterien für die Existenz einer SLTR für $G$ erarbeitet. Diese liefern allerdings nicht sofort einen Algorithmus, weder zur Frage nach der Existenz, noch für das Erlangen einer spezifischen SLTR. Dieses Kapitel wird sich diesem Thema zuwenden und einen von Aerts und Felsner in \cite{af15} erarbeiteten Algorithmus erläutern und analysieren.

\section{SLTRs via Zwei-Fluss}

Das Ziel ist es, für einen gegebenen ebenen intern-3-zusammenhängenden Graphen $G$ mit Aufhängungen $a_1,a_2,a_3$, sowohl einen Schnyder Wood als auch ein FAA jeweils als Lösung eines Fluss-Problems mit einer Quelle und Senke zu erhalten. Diese beiden werden dann in einem Zwei-Fluss-Problem kombiniert, sodass eine zulässige ganzzahlige Lösung ein Ecken-Kompatibles-Paar kodiert und somit eine SLTR resultiert. Wir beschäftigen uns also mit Multi-Fluss-Problemen auf gerichteten Graphen, die wir in Definition \ref{def_multi_flow} eingeführt haben.

\subsection{Schnyder-Wood-Fluss}

Um einen Schnyder Wood als Fluss-Problem zu kodieren, kann man die in Abschnitt \ref{alpha_orientations} eingeführten $\alpha_s$-Orientierungen auf dem Abschluss von $G+G^*$ nutzen. Fusy zeigt in \cite{fusy07} im Zuge der Untersuchung spezifischer $\alpha$-Funktionen, dass sich $\alpha_s$-Orientierungen von $G+G^*$ in linearer Zeit berechnen lassen, sodass wir auch einen Schnyder Wood auf $G$ in linearer Zeit erhalten.

Sei ein ebener intern-3-zusammenhängender Graph $G$ mit Aufhängungen $a_1,a_2,a_3$ gegeben. Machen wir uns an die Konstruktion eines Netzwerks $\mathcal{N}_S$, sodass ein zulässiger ganzzahliger Fluss $\varphi_s$ auf $\mathcal{N}_S$ einer $\alpha_s$-Orientierung von $G+G^*$ entspricht\footnote{Hier stehen die drei \textit{s} jeweils für Schnyder um spätere Verwechslungen zu vermeiden.}. Nach Theorem \ref{alpha_bij} würde die aus $\varphi_s$ erhaltene $\alpha_s$-Orientierung somit auch einen Schnyder Wood auf $G$ induzieren. Besonderes Augenmerk ist hier auf die Möglichkeit einer späteren Kombination mit einem weiteren Netzwerk gelegt, in welchem eine ganzzahlige Lösung einem FAA auf $G$ entspricht, um ein kombiniertes Netzwerk zu erstellen.

Wie schon erwähnt ist $G+G^*$ bipartit, Kanten-Knoten haben Grad 4, Knoten-Knoten haben Grad $\text{deg}(v)$ und Gebiets-Knoten haben Grad $|f|$. Für eine $\alpha_s$-Orientierung muss nach Definition \ref{alpha_bij} A3 jeder Kanten-Knoten Ausgangsgrad 1, jeder Knoten-Knoten Eingangsgrad $\text{deg}(v)-3$ und jeder Gebiets-Knoten Eingangsgrad $|f|-3$ haben. Die Kanten-Knoten am äußeren Gebiet sind in $G+G^*$ immer nach außen orientiert. Somit reicht es, wenn wir nur die inneren Kanten-Kanten $E_{in}$ betrachten und die Orientierung der äußeren Kanten voraussetzten.

\begin{figure}[h]
	\centering
  \includegraphics[width=0.9\textwidth]{schnyder_flow.png}
  \caption{Der Schnyder Wood Fluss durch eine innere Kante $(v,w)$. Die nicht beschrifteten Kanten haben Kapazität 1.}
  \label{schnyder_flow}
\end{figure}

Das im Folgenden konstruierte Netzwerk ist in Abbildung \ref{schnyder_flow} skizziert und unter dem Punkt Netzwerk \ref{net_schnyder} zusammengefasst. Das Netzwerk $\mathcal{N}_S$ hat eine Quelle $s$ und eine Senke $t$ und einen Knoten für jeden Knoten aus $G+G^*$ (bis auf die äußeren Kanten). Es existieren Kanten von der Quelle zu jedem Kanten-Knoten $e \in E_{in}$ mit Kapazität 1. Von den Kanten-Knoten $e$ zu inzidenten Knoten-Knoten $v$ und Gebiets-Knoten $f \in F_{in}$ fügen wir ebenfalls Kanten mit Kapazität 1 ein. Merke, dass der Fluss auf diesen Kanten bei einer ganzzahligen Lösung eine $\alpha_s$-Orientierung auf $G+G^*$ ergeben soll. Nun fügen wir noch Kanten von $f \in F_{in}$ zur Senke mit Kapazitäten $|f|-3$, Kanten von den (inneren) Knoten-Knoten $v \in V_{in} = V \setminus \{a_1,a_2,a_3\}$ zur Senke mit Kapazitäten $\text{deg}(v)-3$ und Kanten von den Aufhängungen $a_i$ zur Senke mit Kapazitäten $\text{deg}(a_i)-2$. Die letzte Kapazität ergibt sich, da die Halbkante in $G+G^*$ von $a_i$ aus immer nach außen orientiert. Wir müssen somit nur noch zwei andere Kanten von $a_i$ weg orientieren. Der Bedarf des Netzwerkes entspricht der Anzahl der inneren Kanten von $G$. Fassen wir zusammen.

\begin{network}[Schnyder Wood]\label{net_schnyder}
Bei $\mathcal{N}_S$ handelt es sich um ein gerichtetes Netzwerk, das auf Basis eines ebenen intern-3-zusammenhängenden Graphen $G$ mit Aufhängungen $a_1,a_2,a_3$ erstellt wird, um einen Schnyder Wood auf $G$ zu finden. Ein Ausschnitt ist in Abbildung \ref{schnyder_flow} dargestellt.
	\begin{itemize}
	\item $\mathcal{N}_S$ hat eine Quelle $s$ und eine Senke $t$
	\item Knoten in $\mathcal{N}_S$ werden für jeden innere Kante $e \in E_{in}$, jedes innere Gebiet $f\in F_{in}$ und jeden Knoten $v \in V$ aus $G$ erzeugt.
	\item Es werden gerichtete Kanten der folgenden Typen in $\mathcal{N}_S$ erzeugt:
		\begin{itemize}
		\item $(s,e)$ von der Quelle zu jeder inneren Kante mit $c\big(s,e\big) = 1$
		\item $(e,v_1),(e,v_2)$ von jeder inneren Kante zu den Endknoten mit $c\big(e,v\big) = 1$
		\item $(e,f)$ von jeder inneren Kante zu adjazenten Gebieten mit $c\big(e,f\big) = 1$
		\item $(v,t)$ von den Knoten zur Senke mit $c\big(v,t\big) = 1$
		\item $(f,t)$ von den inneren Gebieten zur Senke mit $c\big(f,t\big) = |f|-3$
		\item $(a_i,t)$ von den Aufhängungen zur Senke mit $c\big(f,t\big) = \text{deg}(a_i)-2$
		\item $(v,t)$ von den restlichen Knoten zur Senke mit $c\big(f,t\big) = \text{deg}(v)-3$
		\end{itemize}
	\item $\mathcal{N}_S$ hat Bedarf $d=|E_{in}|$
	\item [$\Rightarrow$] Ein zulässiger ganzzahliger Fluss $\varphi_s$ existiert genau dann, wenn ein Schnyder Wood  auf $G$ existiert.
	\end{itemize}
\end{network}

\begin{proposition}
Sei $G$ ein ebener intern-3-zusammenhängender Graph mit Aufhängungen $a_1,a_2,a_3$ und $\mathcal{N}_S$ wie in Netzwerk \ref{net_schnyder} konstruiert. Ein ein zulässiger ganzzahliger Fluss $\varphi_s$ auf $\mathcal{N}_S$ existiert genau dann, wenn ein Schnyder Wood auf $G$ zu den Aufhängungen $a_1,a_2,a_3$ existiert. Insbesondere kodiert jeder Fluss einen Schnyder Wood.
\end{proposition}

\begin{proof}
Sei $\varphi_s$ ein zulässiger ganzzahliger Fluss auf $\mathcal{N}_S$. Es gilt $|\varphi_s| = |E_{in}|$. Somit hat jede innere Kante-Knoten Ausgangsgrad 1. Weiter gilt 
$$\sum_{v \in V} \Big(\text{deg}(v)-3\Big) + 3 + \sum_{f \in F_{in}} \Big(|f|-3\Big) = |E_{in}|.$$
Somit sind die Kanten von den Kanten-Knoten und Knoten-Knoten ausgelastet. Der Fluss $\varphi_s$ entlang einer Auskante von $e \in E_{in}$ in $\mathcal{N}_S$ entspricht dann genau der Kante in $G+G^*$, die in der $\alpha_{s}$-Orientierung von $e$ weg orientiert wird. Durch die Knoten-Knoten $v$ und Gebiets-Knoten $f$ fließt Fluss auf $\text{deg}(v)-3$ bzw. $|f|-3$ Kanten der von den Kanten-Knoten kommt. Somit lässt $\varphi_s$ jeweils drei Kanten frei, denen wir noch keine Orientierung zugewiesen haben. Diese entsprechen somit den von $v$ bzw. $f$ weg orientierten Kanten einer $\alpha_{s}$-Orientierung auf $G+G^*$. Somit hat jedes Gebiet und jeder Knoten den passenden Ausgangsgrad und $\varphi_s$ kodiert eine $\alpha_s$-Orientierung auf $G+G^*$. Auf analoge Weise lässt sich aus einer $\alpha_s$-Orientierung ein zulässiger ganzzahliger Fluss $\varphi_s$ erstellen. Es existiert also genau dann ein Schnyder Wood auf $G$, wenn eine ganzzahlige Lösung $\varphi_s$ für $\mathcal{N}_S$ existiert.
\end{proof}

\subsection{FAA-Fluss}\label{faa-flow}

Sei wieder ein ebener intern-3-zusammenhängender Graph $G$ mit Aufhängungen $a_1,a_2,a_3$ gegeben. Ein FAA ordnet nun jedem Gebiet $|f|-3$ flache Winkel zu und jeder Knoten darf maximal einem Gebiet zugeordnet werden. Die adjazenten Knoten um das äußere Gebiet, die keine Aufhängungen sind, müssen diesem zugewiesen werden. Wir konstruieren ein Netzwerk $\mathcal{N}_F$, für das ein ganzzahliger zulässiger Fluss einem FAA entspricht 

\begin{figure}
	\centering
  \includegraphics[width=0.8\textwidth]{faa_flow.png}
  \caption{Der FAA-Fluss durch einen Winkel $(f,v)$. Die nicht beschrifteten Kanten haben Kapazität 1.}
  \label{faa_flow}
\end{figure}

Das im Folgenden konstruierte Netzwerk ist in Abbildung \ref{faa_flow} skizziert und unter dem Punkt Netzwerk \ref{net_faa} zusammengefasst. Das Netzwerk $\mathcal{N}_F$ hat jeweils eine Quelle und Senke. Wir erstellen in $\mathcal{N}_F$ Knoten für jeden inneren Winkel $(f,v)$, mit $v\in V$ und $f \in F_{in}$, Knoten für alle inneren Gebiete $f \in F_{in}$ und für die inneren Knoten $v\in V_{in}$. Von der Quelle fügen wir Kanten mit Kapazität 1 zu jedem inneren Winkel $(f,v)$, und von jedem inneren Winkel $(f,v)$ jeweils eine Kante zu $f$ und zu $v$ (falls $v\in V_{in}$) mit Kapazität 1 ein. Merke, dass so der Fluss von den Winkeln zu den inneren Knoten einem FAA entspricht. Zuletzt erstellen wir Kanten von jedem inneren Gebiet $f$ zur Senke mit Kapazität 3 und Kanten von jedem Knoten $v$ zur Senke mit Kapazität 1 ein. Der Bedarf des Netzwerks ist $\sum_{f \in F_{in}}|f|$ und entspricht der Anzahl der inneren Winkel von $G$. Fassen wir zusammen.

\begin{network}[FAA]\label{net_faa}
Bei $\mathcal{N}_F$ handelt es sich um ein gerichtetes Netzwerk, das auf Basis eines ebenen intern-3-zusammenhängenden Graphen $G$ mit Aufhängungen $a_1,a_2,a_3$ erstellt wird, um ein FAA zu finden. Ein Ausschnitt ist in Abbildung \ref{faa_flow} dargestellt.
	\begin{itemize}
	\item $\mathcal{N}_F$ hat eine Quelle $s$ und eine Senke $t$
	\item Knoten in $\mathcal{N}_F$ werden für jeden inneren Winkel $(f,v) \in W_{in}$, jedes innere Gebiet $f\in F_{in}$ und jeden Knoten $v \in V$ aus $G$ erzeugt.
	\item Es werden gerichtete Kanten der folgenden Typen in $\mathcal{N}_F$ erzeugt:
		\begin{itemize}
		\item $(s,(f,v))$ von der Quelle zu jedem inneren Winkel mit $c\big(s,(f,v)\big) = 1$
		\item $((f,v),v)$ von jedem inneren Winkel zum Knoten mit $c\big((f,v),v\big) = 1$
		\item $((f,v),f)$ von jedem inneren Winkel zum Gebiet mit $c\big((f,v),f\big) = 1$
		\item $(f,t)$ von den inneren Gebieten zur Senke mit $c\big(f,t\big) = 3$
		\item $(v,t)$ von den Knoten zur Senke mit $c\big(f,t\big) = 1$
		\end{itemize}
	\item $\mathcal{N}_F$ hat Bedarf $d=\sum_{f \in F_{in}}|f|$
	\item [$\Rightarrow$]Ein zulässiger ganzzahliger Fluss $\varphi_F$ existiert genau dann, wenn ein FAA existiert.
	\end{itemize}
\end{network}

\begin{proposition}
Sei $G$ ein ebener intern-3-zusammenhängender Graph mit Aufhängungen $a_1,a_2,a_3$ und $\mathcal{N}_F$ wie in Netzwerk \ref{net_faa} konstruiert. Ein ein zulässiger ganzzahliger Fluss $\varphi_F$ auf $\mathcal{N}_S$ existiert genau dann, wenn ein FAA auf $G$ zu den Aufhängungen $a_1,a_2,a_3$ existiert. Insbesondere kodiert jeder Fluss genau ein FAA.
\end{proposition}

\begin{proof}
Sei $\varphi_F$ ein zulässiger ganzzahliger Fluss auf $\mathcal{N}_F$. Der Fluss auf einer Kante $((f,v),f)$ entspricht einer Ecke von $f$ und Fluss auf $((f,v),u)$ der Zuweisung eines Knoten zu $f$. Zur Vereinfachung sprechen wir im Weitern auch von Ecken- und Zuweisungs-Fluss. Somit wird jeder innere Winkel von $\varphi_F$ entweder einem Gebiet zugewiesen oder als Ecke ausgezeichnet. Da die Kanten von den inneren Knoten zur Senke Kapazität eins haben kann jeder Knoten nur einmal zugewiesen werden. F2 ist somit erfüllt. Von jedem inneren Gebiet $f$ fließt Fluss mir Stärke $|\varphi_s(f,t)| = 3$ zu Senke. Somit muss $|f|-3$ Fluss durch zugewiesene Knoten gehen. F1 ist somit erfüllt. $\varphi_F$ respektiert also die Bedingungen aus Definition \ref{def_faa} und es existieren nur dann FAAs auf $G$, wenn mindestens eine ganzzahlige Lösung für $\mathcal{N}_F$ existiert.
\end{proof}

\begin{remark}

Das oben eingeführte Netzwerk \ref{net_faa} zur Bestimmung von FAAs lässt sich auch als Zwei-Fluss-Problem konstruieren. Wir trennen für Ecken- und Zuweisungs-Fluss die Quellen und Senken.  Zu jedem Winkelknoten $(f,v)$ fügen wir eine Kopie $(f,v)'$ ein und verbinden beide durch eine Kante mit Kapazität 1 (siehe Abbildung \ref{faa_as_2}). In den ersten führt eine Kante von beiden Quellen mit Kapazität 1. Und vom zweiten aus führen Kanten zu $f$ und $v$. Der Bedarf des Ecken-Flusses ist dann $3|F_{in}|$ und der Bedarf des Zuweisung-Flusses $\sum_{f \in F_{in}}{|f|-3}$.

\begin{figure}[h]
	\centering
  \includegraphics[width=0.8\textwidth]{faa_2_flow.pdf}
  \caption{Das einfügen einer Kante und eines zweiten Winkel Knotens gewährt die Trennung von Winkel- und Zuweisungs-Fluss. Die nicht beschrifteten Kanten haben Kapazität 1.}
  \label{faa_as_2}
\end{figure}

Eine zulässige ganzzahlige Lösung $\varphi_F = (\varphi_e,\varphi_z)$ entspricht dann wieder einem FAA auf $G$. Aus der Ganzzahligkeit folgt, dass ein Winkel entweder von $\varphi_{e}$ (Ecke) oder $\varphi_{z}$ (Zuweisung) genutzt wird. Dies gewährleisten die Kanten zwischen den Winkelknoten, da sie immer nur von einem der beiden Flüssen genutzt werden können.
\end{remark}




\subsection{Ein Zwei-Fluss Netzwerk zur Erkennung von SLTRs}

Im Verlauf des Kapitels haben wir nun sowohl für Schnyder Woods als auch für FAAs ein Netzwerk betrachtet, für das eine ganzzahlige Lösung einen Schnyder Wood bzw. ein FAA für einen ebenen Graphen $G$ liefert. Wir wollen jetzt eine Kombination aus beiden erstellen die ein Ecken kompatibles Paar $(\sigma,\phi)$ aus einem Schnyder Labeling $\sigma$ und einem FAA $\phi$  kodiert.\

Es folgt die Konstruktion eines Netzwerkes, wir bezeichnen es mit $\mathcal{N}_G$, welches diesen Wunsch erfüllt. Ein ganzzahliger Fluss $\varphi$ kodiert dann ein Ecken kompatibles Paar und impliziert somit nach Theorem \ref{theo_ccc} eine SLTR für $G$. Es handelt sich hierbei um ein 2-Fluss-Netzwerk.

Wie oben in Abschnitt \ref{faa-flow} erwähnt lässt sich ein FAA auch mit einem Zwei-Fluss kodieren und wir können Ecken- und Zuweisungs-Fluss mit den passenden Bedarfen getrennt betrachten. Wir müssen jetzt diese drei Flüsse, also Schnyder-, Ecken- und Zuweisungs-Fluss in einem Netzwerk kombinieren. In \cite{af15} ergeben Schnyder- und Ecken-Fluss zusammen Fluss von Typ 1 und der Zuweisungs-Fluss Typ 2. Wir wollen hier analog ein Netzwerk konstruieren in dem wir FAA- und Schnyder-Wood Fluss nicht trennen. Der Verständlichkeit wegen werden wir Pfade, die in einer Lösung von einem der drei Flussarten genutzt werden, \textit{Schnyder-Pfad, Ecken-Pfad} und \textit{Zuweisungs-Pfad} nennen.

Bei der Kombination der beiden oben konstruierten Netzwerke $\mathcal{N}_S$ und $\mathcal{N}_F$ zu $\mathcal{N}_G$ müssen die Ecken Kompatibilität von Schnyder Labeling und FAA gewährleistet werden. K1 zu erfüllen, also die Nutzung der gleichen Aufhängungen von $\sigma$ und $\phi$, ist kein Problem. Allerdings müssen wir für die zweite Bedingung das Netzwerk etwas komplizierter machen. Betrachten wir als Basis $\mathcal{N}_S \cup \mathcal{N}_F$ und fürs erste nur das Teilnetzwerk um ein inneres Gebiet $f$, dann sehen wir, dass $f$ in $\mathcal{N}_S$ $|f|-3$ Schnyder-Fluss aufnimmt, aber $|f|$ Einkanten in $\mathcal{N}_S$ hat. Wir können die drei leeren Kanten für den Ecken-Fluss aus $\mathcal{N}_F$ nutzen. Um K2 zu erfüllen, müssen gewährleisten, dass jede Ecke im Schnyder Labeling ein anderes Label hat. Betrachten wir also die von $\varphi_S$ induzierte $\alpha_s$-Orientierung auf dem Abschluss von $G+G^*$. Nach Theorem \ref{alpa_bij} erhalten wir in Bijektion stehende Schnyder Labelings auf $G$ und $G^*$.

\begin{figure}[h]
	\centering
  	\includegraphics[width=0.9\textwidth]{alpha_bij.png}
  	\caption{a) Eine $\alpha_s$-Orientierung um eine innere Kante von $G$. b) Teile der korrespondierenden Schnyder Woods auf $G$ und $G^*$. c) Die induzierten Label, die für $G$ und $G^*$ gleich sind.}
	\label{alpha_bij}
\end{figure}

Für diese gilt, wie in Abbildung \ref{alpha_bij} skizziert, dass das Label der Ecke eines Gebietes in $G$ und das ihr in $G+G^*$ gegenüberliegenden Label der Ecke eines Gebiets um einen Knoten in $G^*$ gleich sind. Für eine zu $v$ in $G^*$ hin orientierte Kante folgt aus der Bijektion zwischen Schnyder Labelings und Schnyder Woods aus Abschnitt \ref{sw}, dass die Label links und rechts am Ende dieser Kante gleich sind. Somit sind auch die Label in $G$ gleich und wir können die folgende Eigenschaft festhalten.
\begin{itemize}
\item [A1] Die Label, des von $\alpha_s$ induzierten Schnyder Wood auf $G$, sind zwischen zwei aufeinander folgenden zu $f$ orientierten Kanten gleich.
\end{itemize}
Da es genau drei zu $f$ orientierte Kanten gibt müssen wir also dafür sorgen, dass für jedes Paar dieser Kanten eine Ecke zwischen ihnen liegt, da so die drei Ecken unterschiedliche Label haben und wir K2 erfüllen. Um dies zu erlangen implementieren wir eine zyklische Struktur um jedes innere Gebiet, wie in Abbildung \ref{combinded_face_sketch} skizziert.\\

\begin{figure}[h]
	\centering
  	\includegraphics[width=0.9\textwidth]{combined_face_sketch.png}
  	\caption{Eine Skizze des kombinierten Netzwerkes auf einem inneren Gebiet mit $|f| = 4$. Beispielhaft sind Schnyder-Fluss (rot), Ecken-Fluss (blau) und Zuweisungs-Fluss (grün) eingezeichnet. }
	\label{combined_face_sketch}
\end{figure}

Betrachten wir zuerst den Schnyder-Fluss. Dieser wird Fluss von Typ 1, also von Quelle 1 zu Senke 1 sein. Für einen Schnyder-Pfad der durch einen Knoten $v$ führt hat sich nichts geändert. Der in der Skizze eingezeichnete Schnyder-Pfad der durch $f$ führt passiert davor einen extra Knoten, wir nennen ihn \textit{kleines Quadrat} der gewährleisten soll, dass von Seite des Gebietes aus entweder ein Schnyder-Pfad oder ein Ecken-Pfad in $f$ mündet. Zuletzt fügen wir wie oben von jedem inneren Gebiet eine Kante mit Kapazität $|f|-3$ zu Senke 1 ein. Somit kodiert hier eine ganzzahlige Lösung weiterhin einen Schnyder Wood auf $G$.\

Kommen wir nun zum FAA-Fluss, also Fluss von Typ 2. Von Quelle 2 geht genau wie in Abbildung \ref{faa_flow} eine Kante zu jedem inneren Winkel $(f,v)$. Ein Zuweisungs-Pfad verlässt diesen Winkel über einen zusätzlich zu $v$ eingefügten Dummy-Knoten $v^*$. Von jedem $v^*$ geht eine Kante mit Kapazität 1 zu einer Dummy-Senke und von dieser eine Kante mit Kapazität $\sum_{f \in F_{in}} |f|-3$ zu Senke 2, wie in Abbildung \ref{dummy_sink} illustriert.

Die Dummy-Knoten sorgen dafür, dass jeder Knoten im FAA nur einmal zugewiesen werden kann, ohne in Konflikt mit dem Schnyder-Fluss zu kommen. Die eingeschobene Dummy-Senke beschränkt die Anzahl der zugewiesenen Knoten, genau wie im zuvor konstruierten FAA-Fluss, auf $\sum_{f \in F_{in}} |f|-3$.

\begin{figure}[h]
	\centering
  	\includegraphics[width=0.7\textwidth]{dummy_sink.png}
  	\caption{Der Zuweisungsfluss durch die Winkel, Dummy-Knoten und die zusätzliche Kante vor Senke 2. Die Kante rechts hat Kapazität $\sum_{f \in F_{in}} |f|-3$ und alle anderen Kapazität 1.}
	\label{dummy_sink}
\end{figure}

Es bleibt der Ecken-Fluss. Hier betritt der Pfad das Gebiet $f$ wieder durch einen Winkel und muss es über ein ungenutztes kleines Quadrat verlassen. Die zweite und dritte Kante in jedem Winkeldreieck gewährleisten, dass nicht immer das nächste kleine Quadrat genutzt werden muss. Falls dies von Schnyder-Fluss besetzt ist und der nächste Winkel zugewiesen wird, kann ein Ecken-Pfad den nächsten Winkel passieren. Weiterhin sorgt die erste Kante, die von sowohl Schnyder-, als auch Winkel-Pfaden genutzt werden kann, für eine eindeutige Beschriftung (als Ecke oder nicht) im Falle einer ganzzahligen Lösung. Wie oben existieren auch hier Kanten von jedem inneren Gebiet zu Senke 2 mit Kapazität drei.

Betrachten wir die Bedarfe der beiden Flüsse von Typ 1 und Typ 2, $\varphi_1$ bzw. $\varphi_2$. Beide entsprechen jeweils den Bedarfen der oben konstruierten $\mathcal{N}_S$ und $\mathcal{N}_F$, da mit den gleichen Argumenten wie oben, ein Schnyder Wood und ein FAA kodiert werden können. Jedes Gebiet benötigt genau drei Ecken und $|f|-3$ zugewiesene Knoten und je ein Schnyder-Pfad führt durch jede innere Kante, $|E_{in}|$. Hier seien wieder $E_{in}$ die inneren Kanten und $F_{in}$ die inneren Gebiete von $G$. Es gilt also:

\begin{itemize}
\item $d_1$ = Bedarf$(\varphi_1) = $ Bedarf$(\varphi_S) = |E_{in}|$
\item $d_2$ = Bedarf$(\varphi_2) = $ Bedarf$(\varphi_F) =  \sum_{f \in F_{in}}(|f|-3) + 3|F_{in}| = \sum_{f \in F_{in}} |f|$
\end{itemize}

Bevor wir in Theorem \ref{theo_algo} zeigen, dass eine ganzzahlige Lösung $\varphi=(\varphi_1,\varphi_2)$ auch wirklich ein Ecken kompatibles Paar kodiert, wollen wir noch ein Paar weitere Beobachtungen festhalten. Nehmen wir also an, wir haben eine ganzzahlige Lösung $\varphi$ gefunden, dann gilt für diese:
\begin{itemize}
\item [A2] Jede äussere Kante in einem Winkel-Dreieck ist ausgelastet, sie wird entweder von einem Ecken- oder Zuweisungspfad genutzt.
\item [A3] Jede Kante von einem kleinen Quadrat zu einem inneren Gebiet $f$ ist ausgelastet, sie wird entweder von einem Schnyder- oder Ecken-Pfad genutzt.
\item [A4] Ein inneres Gebiet $f$ mit $|f|=3$ kann nicht von Zuweisungs- bzw. Schnyder-Pfaden genutzt werden.
\end{itemize}

Wir wollen diese Beobachtungen kurz begründen. Für jede mögliche ganzzahlige Lösung $\varphi$ gilt $$|\varphi|=|\varphi_1|+|\varphi_2| = |E_{in}| + \sum_{f \in F_{in}} |f|.$$
Da es genau $\sum_{f \in F_{in}} |f|$ innere Winkel gibt und der FAA-Fluss $\mathcal{N}_G$ nur durch diese betreten kann ergibt sich A2. A3 wird aus Gleichung \ref{eq_sat} weiter unten folgen. Durch ein inneres Gebiet $f$ müssen drei Ecken-Pfade führen und im Fall $|f|=3$ führt dies zu A4, da kein Platz in den Winkeln für Zuweisungs-Pfade und keine freien kleinen Quadrate für Schnyder-Pfade existieren.

\begin{theorem}\label{theo_algo}
Sei $G$ ein intern-3-zusammenhängender Graph mit gegebenen Aufhängungen $\{a_1,a_2,a_3\}$, dann existiert eine SLTR von $G$, genau dann wenn ein ganzzahliger zulässiger Fluss $\varphi=(\varphi_1,\varphi_2)$ auf $\mathcal{N}_G$ existiert.
\end{theorem}

Fassen wir vor dem Beweis noch einmal das Netzwerk zusammen.

\begin{network}[SLTR]\label{net_sltr}
Bei $\mathcal{N}_G$ handelt es sich um ein gerichtetes Netzwerk, das auf Basis von $G$ erstellt wird, um eine SLTR von $G$ zu finden. Ein Ausschnitt um ein inneres Gebiet ist in Abbildung \ref{theo_algo} dargestellt.
	\begin{itemize}
	\item $\mathcal{N}_G$ hat zwei Quellen $s_1,s_2$ und zwei Senken $t_1,t_2$
	\item Knoten in $\mathcal{N}_s$ werden für jeden innere Kante $e \in E_{in}$, jedes innere Gebiet $f\in F_{in}$ und jeden Knoten $v \in V$ aus $G$ erzeugt.
	\item Es werden Knoten der folgenden Typen in $\mathcal{N}_S$ erzeugt:
		\begin{itemize}
		\item Knoten $e$ für jede innere Kante $e \in E_{in}$
		\item Knoten $v$ für jeden $v \in V$ und Dummy-Knoten $v^*$ für jeden $v \in V_{in}$
		\item Knoten für jedes innere Gebiet $f$.
		\item $|f|$ kleine Quadrate $q$ um jedes innere Gebiet $f$
		\item Vier Knoten $w_1,w_2,w_3,w_4$ für jedes innere Winkeldreieck
		\item Die Dummy-Senke $t_d$
		\end{itemize}
	\item Es werden gerichtete Kanten der folgenden Typen in $\mathcal{N}_G$ erzeugt:
		\begin{itemize}
		\item $(s_1,e)$ von Quelle 1 zu jeder inneren Kante mit $c\big(s,e\big) = 1$
		\item $(e,v_1),(e,v_2)$ von jeder inneren Kante zu den Endknoten mit $c\big(e,v\big) = 1$
		\item $(e,q)$ von inneren Kanten zu adjazenten kleinen Quadraten mit $c\big(e,q\big) = 1$
		\item $(q,f)$ von jeder kleinen Quadraten zu den inneren Gebieten mit $c\big(q,f\big) = 1$
		\item $(f,t_1)$ von den inneren Gebieten zur Senke 1 mit $c\big(f,t_1\big) = |f|-3$
		\item $(a_i,t_1)$ von den Aufhängungen zur Senke 1 mit $c\big(f,t\big) = \text{deg}(a_i)-2$
		\item $(v,t_1)$ von den restlichen Knoten zur Senke 1 mit $c\big(f,t\big) = \text{deg}(v)-3$		
		\item $(s_2,(f,v))$ von Quelle 2 zu jedem inneren Winkel mit $c\big(s_2,(f,v)\big) = 1$
		\item $(w_1,w_2),(w_2,w_3),(w_3,w_4)$ in jedem inneren Winkel mit $c\big(w_i,w_{i+1}\big) = 1$
		\item $(t_4,q)$ von inneren Winkeln zum nächsten kleinen Quadrat mit $c\big(t_4,q\big) = 1$
		\item $(t_4,t'_3)$ von inneren Winkeln zum nächsten inneren Winkel mit $c\big(t_4,t'_3\big) = 1$
		\item $((f,v),f)$ von inneren Winkeln zum Gebiet mit $c\big((f,v),f\big) = 1$
		\item $(t_2,v*)$ von jedem inneren Winkel zum Dummy-Knoten mit $c\big(t_2,v^*\big) = 1$
		\item $(v^*,t_d)$ von den Dummy-Knoten zur Dummy-Senke mit $c\big(f,t\big) = 1$
		\item $(t_d,t_2)$ von der Dummy-Senke zu Senke 2 mit $c\big(t_d,t_2\big) = \sum_{f \in F_{in}}|f|-3$
		\end{itemize}
	\item $\mathcal{N}_S$ hat Bedarfe $d_1=|E_{in}|$ und $d_2 = \sum_{f \in F_{in}}|f|$
	\item [$\Rightarrow$] Ein zulässiger ganzzahliger Fluss $\varphi = (\varphi_1,\varphi_2)$ existiert. $\Leftrightarrow$ Es existiert ein SLTR  auf $G$.
	\end{itemize}
\end{network}	

\begin{proof}[Beweis von Theorem \ref{theo_algo}]
Sei $G$ ein intern-3-zusammenhängender Graph mit Aufhängungen $\{a_1,a_2,a_3\}$ und $\varphi=(\varphi_1,\varphi_2)$ sei ein ganzzahliger machbarer Fluss auf $\mathcal{N}_G$. Im ersten Schritt extrahieren wir einen Schnyder-Wood $\sigma$ und ein FAA $\phi$, um dann zu zeigen, dass sie ein Ecken kompatibles Paar bilden. Für einen machbaren Fluss müssen die Bedarfe erfüllt werden. Es gilt somit $|\varphi_1| =  |E_{in}|$ und $|\varphi_2| = \sum_{f \in F_{in}} |f|$.
\begin{equation}\label{eq_sat}
\begin{split}
|\varphi_1| + |\varphi_2| & = \sum_{f \in F_{in}} (|f|-3) + 3|F_{in}| + |E_{in}|\\
		& = \sum_{f \in F_{in}} (|f|-3) + 2|E| -|V| - 1 + 2|F| - |f_{aus}|\\
		& = \sum_{f \in F_{in}} (|f|-3) + \sum_{v \in V} (\text{deg}(v)-3) + 2|V| + 2|F| - 1 - |f_{aus}|\\
		& = \sum_{f \in F_{in}} (|f|-3) + \sum_{v \in V} (\text{deg}(v)-3) + 2|E| + 3 - |f_{aus}|\\
		& = \underbrace{\sum_{f \in F_{in}}(|f|-3)  }_{\text{\parbox{8em}{Dummy-Senke zu Senke 2}}} + \underbrace{\sum_{v \in V} (\text{deg}(v)-3) +3 }_{\text{\parbox{10em}{Kapazität Senke 2 von den Knoten.}}} +\underbrace{\sum_{f \in F_{in}}(|f|-3) + 3|F_{in}|}_{\text{\parbox{12em}{Kanten von den Quadraten zu den inneren Gebieten}}}
\end{split}
\end{equation}

Die beiden Terme in der rechten unteren Klammer entsprechen den Kapazitäten von den inneren Gebieten zu Senke 1 und Senke 2. Somit sind alle Kanten zu den Senken ausgelastet. Die Kanten von den kleinen Quadraten zu den inneren Gebieten sind ebenfalls ausgelastet. Diese sind die einzigen Kanten in $\mathcal{N}_G$, die sowohl von $\varphi_1$ als auch $\varphi_2$ genutzt werden können. Kapazität eins und Ganzzahligkeit von $\varphi$ impliziert somit A3.

Beginnen wir mit $\varphi_1$ um einen Schnyder Wood, oder genauer eine $\alpha_s$-Orientierung, zu erhalten. $|\varphi_1| = |E_{in}|$, somit führt durch jede innere Kante ein Schnyder-Pfad und dieser gibt uns die nach aussen gerichtete Kante in $\alpha_s$. Es bleibt zu zeigen, dass für jedes innere Gebiet und jeden Knoten die Bedingungen aus Theorem \ref{alpha_bij} für eine $\alpha_s$ eingehalten werden. Da alle Kanten von den Knoten zu Senke 1 ausgelastet sind folgt, dass durch jeden inneren Knoten $v$ genau $\text{deg}(v)-3$ Schnyder-Pfade führen. Somit ergeben die leeren Einkanten von $v$ in $\mathcal{N}_G$ die drei Auskanten für $\alpha_s$. Für eine Aufhängung $a_i$ folgt analog, dass die beiden ungenutzten Einkanten, zusammen mit der Halbkante ins äußere Gebiet, die Bedingungen der $\alpha_s$-Orientierung erfüllen. Es bleibt zu zeigen, dass durch jedes innere Gebiet $|f|-3$ Schnyder-Pfade führen. Der restliche Schnyder-Fluss $|E_{in}| - \sum_{v \in V} (\text{deg}(v)-3)$ muss durch die inneren Gebiete führen und aus der ersten und letzten Zeile von Gleichung \ref{eq_sat} folgt $$|E_{in}| - \sum_{v \in V} (\text{deg}(v)-3) = \sum_{f \in F_{in}} (|f|-3).$$
Somit führen $|f|-3$ Schnyder-Pfade durch jedes innere Gebiet und wir können die $\alpha_s$-Orientierung vervollständigen und erhalten einen Schnyder Wood auf $G$.\\

Betrachten wir nun $\varphi_2$. Nach A4 sind alle äusseren Kanten in den Winkeln ausgelastet. Falls diese nun in jedem inneren Gebiet von drei Ecken-Pfaden und $|f|-3$ Zuweisungs-Pfaden genutzt werden, können wir ein FAA extrahieren. Da alle Kanten zu Senke 2 ausgelastet sind, führen $\sum_{f \in F_{in}} (|f|-3)$ Pfade durch die Dummy-Senke. Somit werden auch $\sum_{f \in F_{in}} (|f|-3)$ Knoten inneren Gebieten zugewiesen. Indem wir die Pfade zurückverfolgen und sehen aus welchem Gebiet der Zuweisungs-Pfad einen Dummy-Knoten betritt, können wir diese Informationen auslesen. Es bleibt zu zeigen, dass jedem Gebiet genau $|f|-3$ Knoten zugewiesen werden. Dies gilt, wenn durch jedes Gebiet drei Ecken-Pfade laufen und folgt somit, da die Kanten von den inneren Gebieten zu Senke 2 ausgelastet sind. Wir können also aus $\varphi_2$ ein FAA für $G$ extrahieren. \\

Nun müssen wir zeigen, dass $\sigma$ und $\phi$ ein Ecken kompatibles Paar ergeben. C1, dass beide die gleichen Aufhängungen nutzen folgt sofort aus der Konstruktion von $\mathcal{N}_G$. Es bleibt C2.\

Betrachten wir ein Teilnetzwerk (wie in Abbildung \ref{combined_face_sketch}) um ein inneres Gebiet $f$. Die drei Ecken-Pfade können keine der $|f|-3$ kleinen Quadrate nutzen die schon von Schnyder-Fluss okkupiert werden. Die drei übrigen kleinen Quadrate nennen wir \textit{verfügbar}. Ausgehend von $f$ folgen wir den Ecken-Pfaden rückwärts zu den verfügbaren kleinen Quadraten. Wenn wir das Quadrat verlassen gelangen wir zur dritten Kante eines Winkeldreiecks (entgegen dem Uhrzeigersinn). Nun verlassen wir das Gebiet entweder über diesen Winkel oder bewegen uns weiter (entgegen dem Uhrzeigersinn) zum nächsten Winkeldreieck. Doch wir werden zeigen, dass dies nur dann geschieht wenn das kleine Quadrat zwischen diesen nicht \textit{verfügbar} ist. Also betritt zwischen zwei \textit{verfügbaren} kleinen Quadraten ein Ecken-Pfad das Gebiet und die Winkel haben nach A1 unterschiedliche Label.

\begin{claim}\label{claim1}
Seien $Q_1,Q_2$ und $Q_3$, im Uhrzeigersinn, die drei verfügbaren kleinen Quadrate um ein inneres Gebiet $f$. Dann existiert ein Ecken-Pfad, welcher das Netzwerk über $Q_i$ verlässt. Dieser betritt es in einem Winkel zwischen, im Uhrzeigersinn, $Q_{i-1}$ und $Q_i$.
\end{claim}

Angenommen dies ist nicht der Fall und nehmen wir ohne Beschränkung der Allgemeinheit an, dass der Ecken-Pfad $P_e$ das Gebiet durch $Q_3$ verlässt. Der Winkel über den $P_e$ das Teilnetzwerk um das innere Gebiet betritt liegt also nicht zwischen $Q_2$ und $Q_3$. Angenommen er liegt zwischen $Q_1$ und $Q_2$. Betrachte das letzte Winkeldreck vor $Q_2$. Nach unserer Annahme ist die innere Kante dieses Dreiecks von $P_e$ ausgelastet. Somit kann kein Ecken-Fluss zu $Q_2$ gelangen und wir erhalten einen Widerspruch, da alle kleinen Quadrate entweder von Ecken- oder von Schnyder-Fluss genutzt werden müssen. Mit dem gleichen Argument kann $P_e$ das Teilnetzwerk nicht zwischen $Q_3$ und $Q_1$ betreten. Somit ist Behauptung \ref{claim1} wahr.

\begin{claim}
Alle Winkel zwischen zwei aufeinander folgenden verfügbaren kleinen Quadraten, haben die selben Label im Schnyder Labeling $\sigma$.
\end{claim}
Diese Behauptung folgt aus der in Abbildung \ref{alpha_bij} illustrierten Bijektion zwischen der $\alpha_S$ Orientierung und den Schnyder Labelings auf $G$ und $G^*$. Die Winkel links und rechts von einem kleinen Quadrat, dass von einem Schnyder-Pfad genutzt wird, haben das gleiche Label in $\sigma$, da diese den Einkanten in $\alpha_s$ entsprechen. Die Auskanten entsprechen den verfügbaren kleinen Quadraten, und hier ändern sich die Label.\\

Diese beiden Behauptungen zusammen zeigen, dass jede Ecke aus $\phi$ ein anderes Label in $\sigma$ hat. Somit handelt es sich um ein Ecken Kompatibles Paar $(\sigma,\phi)$.\\

Wir haben die Rückrichtung gezeigt. Nehmen wir also an, dass eine SLTR für $G$ existiert. Wir müssen nun einen zulässigen ganzzahligen Fluss $\varphi=(\varphi_1,\varphi_2)$ auf $\mathcal{N}_G$ konstruieren, der die SLTR kodiert. Nach Theorem \ref{theo_ccc} existiert ein Ecken kompatibles Paar $(\sigma,\phi)$ aus einem Schnyder Labeling $\sigma$ und einem FAA $\phi$, das zu diesem SLTR passt. Betrachte die zu $\sigma$ gehörige $\alpha_s$-Orientierung.

Wir beginnen mit einem leeren, wie oben konstruierten Netzwerk $\mathcal{N}_G$ und werden nun Schritt für Schritt einen zulässigen Fluss $\varphi$ konstruieren.

Zuerst fügen wir für jeden zugewiesenen Winkel einen Pfad von Quelle 2, über die äussere Kante des Winkeldreiecks, den zugehörigen Dummy-Knoten und die Dummy-Senke hin zu Senke 2 ein. Es kommen somit $\sum_{f \in F_{in}}|f|-3$ Einheiten Fluss hinzu und die Kante von der Dummy-Senke zu Senke 2 wird ausgelastet.

Als nächsten fügen wir den Fluss hinzu, der die $\alpha_s$-Orientierung kodiert. Zuerst von Quelle 1 zu jedem inneren Kanten-Knoten $e$, dann von den inneren Kanten entweder über ein kleines Quadrat in ein angrenzendes Gebiet oder zu einem benachbarten Knoten je nachdem, wohin die Auskante von $e$ in $\alpha_s$ zeigt. Zuletzt saturieren wir die Kanten von den inneren Knoten und inneren Gebieten zu Senke 1.

Zuletzt müssen wir den Ecken-Fluss einfügen. Ein Ecken-Pfad $P_e$ entspringt in Quelle 1, nutzt das zugehörige Winkeldreieck (diese sind noch frei) und verlässt das Gebiet über das im Uhrzeigersinn nächste verfügbare kleine Quadrat, wieder. 

Es sind alle Kanten hin zu den Senken ausgelastet. Ebenso kann man sehen, dass an keiner Kante die Kapazität überschritten wird. Somit haben wir einen zulässigen ganzzahligen Fluss kostruiert, der eine SLTR kodiert. Damit ist der Beweis abgeschlossen.

\end{proof}

\section{Nicht ganzzahlige Lösungen}

Wir werden uns zum Abschluss des Kapitels mit der von Aerts und Felsner offen gelassenen Frage beschäftigen, ob die Erkennung von Graphen mit einer SLTR in P liegt -- ob also polynominelle Algorithmen für die Entscheidung existieren, dass für einen gegebenen Graphen eine SLTR existiert. Wie in Abschnitt \ref{dir_multi_flow} erwähnt, impliziert eine nicht ganzzahlige Lösung für ein Gerichtetes-Multi-Fluss-Problem auf einem Graphen mit $n\geq 2$ Paaren von Quellen und Senken, im Allgemeinen nicht die Existenz einer ganzzahligen Lösung und nach Theorem \ref{np_hard} ist die Berechnung einer ganzzahligen Lösung NP-schwer. Die experimellen Ergebnisse aus Kapitel \ref{the_program} lassen jedoch die Möglichkeit offen, dass wir für das betrachtete Netzwerk $\mathcal{N}_G$ die folgende Vermutung beweisen können.

\begin{conjecture}\label{int_conj}
Sei $\tilde{\varphi}=(\tilde{\varphi_1},\tilde{\varphi_2})$ ein nicht ganzzahliger zulässiger Fluss auf $\mathcal{N}_G$, dann existiert auch ein ganzzahliger zulässiger Fluss $\varphi$ und wir können in polynomieller Zeit ein Gutes-FAA aus $\tilde{\varphi_2}$ konstruieren.
\end{conjecture}

\begin{remark}
Wenn wir nicht darauf bestehen, dass ein berechneter zulässiger Fluss auf $\mathcal{N}_G$ ganzzahlig ist, dann lässt sich ein zulässiger Fluss wie in Abschnitt \ref{dir_multi_flow} erwähnt durch lineare Programmierung in polynomineller Zeit finden und das Entscheidungsproblem, ob ein Graph eine SLTR hat läge so in P.
\end{remark}

Es ist uns noch nicht gelungen, einen Beweis von Vermutung \ref{int_conj} zu finden. Wir werden zuerst einen Weg besprechen, um ein FAA aus einer nicht-ganzzahligen Lösung zu konstruieren. Dieses FAA wäre dann nach Vermutung \ref{int_conj} ein Gutes-FAA. Im weiteren Verlauf dieses Abschnittes betrachten wir dann ein Paar der möglichen und verfolgten Beweisansätze. Um die Argumentation einfacher zu gestalten, werden wir das Zwei-Fluss-Problem manchmal als Drei-Fluss Problem, mit einer Lösung $\varphi=(\varphi_s,\varphi_e,\varphi_z)$, betrachten, indem wir Schnyder-, Ecken-, und Zuweisungs-Fluss eigene Typen und somit Quellen und Senken zuweisen.

\begin{proposition}
Sei $\mathcal{N}^*_G$ ein Netzwerk bei dem wir Schnyder-, Ecken-, und Zuweisungs-Fluss jeweils eigene Quellen und Senken zuweisen und das sonst analog zu $\mathcal{N}_G$ konstruiert wird. Dann existiert ein zulässiger (ganzzahliger) Fluss $\varphi^*=(\varphi^*_s,\varphi^*_e,\varphi^*_z)$ auf $\mathcal{N}^*_G$ genau dann, wenn ein zulässiger (ganzzahliger) Fluss $\varphi=(\varphi_1,\varphi_2)$ auf 
$\mathcal{N}_G$ existiert.
\end{proposition}

\begin{proof}
Wir müssen den FAA-Fluss aus $\mathcal{N}_G$ in zwei Typen aufteilen. In der Bemerkung nach Proposition \ref{prop_net_faa} haben wir schon einen Weg beschrieben dies zu tun. In $\mathcal{N}_G^*$ existieren jeweils von der Ecken- und Zuweisungs-Quelle Kanten zum ersten Knoten der Winkeldreiecke. Die Winkel-Senke ist Senke 2 und wir trennen die Dummy-Senke von Senke 2 und machen sie zur Zuweisungs-Senke. Sonst sind $\mathcal{N}_G$ und $\mathcal{N}^*_G$ gleich. Die Bedarfe werden entsprechend aufgeteilt. Somit sind die einzigen von Ecken- und Zuweisungs-Fluss gemeinsam nutzbaren Kanten die ersten Kanten in den Winkeldreiecken. Für den Schnyder-Fluss verändert sich nichts.

Angenommen wir haben einen zulässigen (ganzzahligen) Fluss $\varphi^*=(\varphi^*_s,\varphi^*_e,\varphi^*_z)$ auf $\mathcal{N}_G^*$, dann induziert $(\varphi_s,\varphi_e+\varphi_z)$ auch einen zulässigen (ganzzahligen) Fluss auf $\mathcal{N}_G$. Sei nun $\varphi=(\varphi_1,\varphi_2)$ ein zulässiger (ganzzahliger) Fluss auf $\mathcal{N}_G$. Dann gilt $\varphi_1 = \varphi^*_s$. Wir müssen also $\varphi_2$ aufteilen. Ein (ganzzahliger) FAA-Pfad der ein Winkeldreieck betritt verlässt es entweder durch einen Dummy-Knoten oder führt weiter in das Gebiet. Wir können $\varphi_2$ somit in $\varphi_e^*$ und $\varphi_z^*$ aufteilen. Es bleibt zu zeigen, dass die Bedarfe erfüllt sind. In $\mathcal{N}_G$ führt eine Kante mit Kapazität drei von jedem inneren Gebiet zu Senke 2, die von $\varphi_2$ gesättigt ist. Dieser Anteil des FAA-Flusses wird nach Konstruktion zu $\varphi_e^*$. Somit bleibt in jedem inneren Gebiet $|f|-3$ Zuweisungs-Fluss. Die Bedarfe sind somit erfüllt. Die Ecken-Kompatibilität ergibt sich wie im Beweis zu Theorem \ref{theo_algo}.
\end{proof}

Insbesondere gilt dieser Zusammenhang für eine beliebige Kombination von Ecken-, Schnyder- und Zuweisungs-Fluss zu zwei Flüssen, und wir müssen die Netzwerke nur leicht abwandeln.

\begin{proposition}\label{choose_types}
Für jede beliebige Kombination von Schnyder-, Ecken- und Zu\-weis\-ungs-Fluss als zwei Flüsse auf einem zu $\mathcal{N}^*_G$ analog konstruierten Netzwerk hat existiert ein zulässiger (ganzzahliger) Fluss $\varphi$ genau dann, wenn ein zulässiger (ganzzahliger) Fluss $\varphi^*=(\varphi^*_s,\varphi^*_e,\varphi^*_z)$ auf $\mathcal{N}^*_G$ existiert. Ein zulässiger (ganzzahliger) Fluss $\varphi$ existiert also insbesondere genau dann, wenn eine zulässige (ganzzahlige) Lösung für eine beliebige andere Kombination der Flüsse existiert.
\end{proposition}

\begin{proof}
Wie wir einen zulässigen (ganzzahligen) Fluss $\varphi$ aus $\varphi^*$ konstruieren ist klar. Betrachten wir denn Fall, dass wir Schnyder- und Zu\-weis\-ungs-Fluss zu $\varphi_1$ kombinieren. Wir fügen zwischen der Zuweisungs-Quelle und den Winkeldreiecken für jedes innere Gebiet einen Beutel ein, zu dem eine Kante mit Kapazität $|f|-3$ von der Zuweisungs-Quelle führt (vergleiche Abbildung \ref{pic_faa_choice}). Sei nun $\varphi=(\varphi_1,\varphi_2)$ ein zulässiger (ganzzahlige) Fluss. Für den Zuweisungs-Fluss $\varphi_2=\varphi_z$ verändert sich nichts. Nun bleiben in jedem inneren Gebiet genau drei Winkeldreiecke von $\varphi_z$ ungenutzt. Wir können einen Pfad in $\varphi_1$ Zuweisungs-Fluss oder Schnyder-Fluss zuweisen, indem wir überprüfen, ob er das kleine Quadrat von einem Kanten-Knoten oder einem Winkeldreieck aus betritt. Durch jedes innere Gebiet führen drei Ecken-Pfade und somit $|f|-3$ Schnyder-Pfade. Die Ecken-Kompatibilität erfolgt wieder analog zum Beweis von Theorem \ref{theo_algo}. Die Kombination aus Schnyder- und Ecken-Fluss folgt nach Aerts und Felsner \cite{af15}. Hier reicht wieder der Beutel im Zuweisungs-Fluss um die Korrektheit des Netzwerkes zu erhalten.
\end{proof}

\begin{claim}
Angenommen es existiert eine zulässige nicht-ganzzahlige Lösung $\tilde{\varphi}$ auf $\mathcal{N}_G$ und $\mathcal{N}_G$ lässt keine ganzzahlige Lösung zu, dann müssen die drei Flüsse $\tilde{\varphi}_e, \tilde{\varphi}_z$ und $\tilde{\varphi}_s$ alle nicht-ganzzahlig sein. Insbesondere gilt dies schon auf jedem Teilnetzwerk um ein inneres Gebiet.
\end{claim}

Angenommen nicht, dann könnten wir lokal den ganzzahligen Fluss als Typ 1 und die beiden anderen als Typ 2 definieren. Nach Theorem \ref{theo_int_flow} existiert nun ein zulässiger ganzzahliger Fluss von Typ 2 und wir würden einen ganzzahligen Fluss auf $\mathcal{N}_G$ erhalten. 

Betrachten wir zunächst den zweiten Teil von Vermutung \ref{int_conj}. Die nächste Proposition liefert eine Methode aus einem nicht ganzzahligen zulässigen Fluss ein FAA zu konstruieren.

\begin{proposition}\label{lem_faa}
Sei $\tilde{\varphi}$ ein nicht ganzzahliger zulässiger Fluss auf $\mathcal{N}_G$ und sei $W$ die Menge der vom Zuweisungsfluss $\tilde{\varphi}_z$ genutzten inneren Winkel von $G$. Dann existiert eine Teilmenge $\phi\subseteq W$, sodass aus jedem Gebiet $f$ genau $|f|-3$ Winkel in $\phi$ enthalten sind und in der jeder Knoten $v$ höchstens einmal vorkommt. $\phi$ ist also ein FAA auf $G$.
\end{proposition}

\begin{proof}
Sei $\tilde{\varphi}$ ein nicht ganzzahliger zulässiger Fluss auf $\mathcal{N}_G$. Wir definieren das gerichtete Netzwerk $\mathcal{F}_z$ mit Quelle $s$ und Senke $t$, einem Beutel $B_f$ für jedes innere Gebiet $f$, einem Knoten für jeden inneren Winkel $(f,v) \in W$ und einem Knoten für jeden Dummy-Knoten. Zuerst fügen Kanten mit Kapazität $|f|-3$ von der Quelle zu jedem Beutel ein. Dann folgen Kanten von den Beuteln $B_f$ zu den Winkeln von $f$, von den Winkeln $(f,v)$ zu den Dummy-Knoten $v^*$ und zuletzt eine Kante von jedem Dummy-Knoten zu Senke, jeweils mit Kapazität 1. Die Kanten zu den Beuteln bilden einen Schnitt mit Kapazität $\sum_{f \in F_{in}}(|f|-3)$ und $\tilde{\varphi}_z$ induziert einen zulässigen nicht-ganzzahligen Fluss $\varphi^*$ auf $\mathcal{F}_z$. Die Stärke eines maximalen $s$-$t$-Fluss in $\mathcal{F}_z$ ist somit $\sum_{f \in F_{in}}(|f|-3)$. Nach Theorem \ref{theo_int_flow} existiert also ein ganzzahliger zulässiger Fluss $\varphi$ auf $\mathcal{F}_z$, mit $|\varphi| = |\varphi^*| = |\tilde{\varphi}_z| = \sum_{f \in F_{in}}(|f|-3).$

\begin{figure}
	\centering
  	\includegraphics[width=0.9\textwidth]{lem_faa_choice.png}
  	\caption{Skizze des Netzwerkes $\mathcal{F}_z$. Die Kanten von der Quelle zu einem Beutel $B_f$ hat Kapazität $|f|-3$ und alle anderen haben Kapazität 1.}
	\label{pic_faa_choice}
\end{figure}

$\varphi$ weißt nun jedem inneren Gebiet $f$ genau $|f|-3$ Winkel zu und jeder Knoten $v$ kann nur einmal zugewiesen werden. Wenn wir noch die per Konstruktion von $\mathcal{N}_G$ zugewiesenen Knoten am äußeren Gebiet hinzunehmen, dann erhalten wir ein FAA auf $G$.
\end{proof}

Wenn wir zeigen könnten, dass ein wie in Proposition \ref{lem_faa} konstruiertes $\phi$ ein Gutes-FAA ist, folgt Vermutung \ref{int_conj}, da die Existenz eines Guten-FAAs $\phi$ nach Theorem \ref{theo_algo} auch die Existenz eines ganzzahligen zulässigen Flusses $\varphi$ für $\mathcal{N}_G$ impliziert. Die Ergebnisse aus Kapitel \ref{the_program} legen nahe, dass es sich um ein Gutes-FAA handelt. Formulieren wir dies als eine zweite Vermutung.

\begin{conjecture}\label{conj_faa}
Ein wie in Proposition \ref{lem_faa}, aus einem zulässigen Fluss auf $\mathcal{N}_G$ konstruiertes FAA $\phi$ ist ein Gutes-FAA von $G$ und induziert somit eine SLTR.
\end{conjecture}

\begin{example}
Es ist uns nicht möglich mit beliebigen Winkeln aus $W$ zu beginnen und Schritt für Schritt für jedes Gebiet $|f|-3$ Winkel wählen. Betrachte den planaren Graphen $G$ aus Abbildung \ref{lem_faa_choice_ex}. Die beiden SLTRs auf der linken Seite haben die selben Aufhängungen, implizieren jedoch andere FAAs somit auch andere zulässige ganzzahlige Flüsse auf $\mathcal{N}_G$. Seien $\varphi$ und $\varphi'$ diese Flüsse und $f_{r},f_{g}$ und $f_b$ die drei eingefärbten Gebiete. Betrachten wir die Zuweisungs-Flüsse $\varphi_z$ und $\varphi'_z$. Dann gilt $$|\varphi_z(f_r,v)|=|\varphi_z(f_r,u)|=|\varphi_z(f_b,w)| = |\varphi_z(f_g,x)| = 1$$
$$|\varphi_z(f_r,v)|=|\varphi_z(f_r,w)|=|\varphi_z(f_b,u)| = |\varphi_z(f_g,x)| = 1.$$
Der Fluss $\tilde{\varphi}=\frac{\varphi+\varphi'}{2}$ ist ebenfalls zulässig und es folgt:
$$|\tilde{\varphi}_z(f_r,v)|=|\tilde{\varphi}_z(f_g,x)| = 1 \text{ und } |\tilde{\varphi}_z(f_r,w)|=|\tilde{\varphi}_z(f_r,u)| = |\tilde{\varphi}_z(f_b,w)|=|\tilde{\varphi}_z(f_b,u)| = \frac{1}{2}.$$
Somit liegen all diese Winkel in W. Wir können allerdings nicht einfach beginnen in einem Gebiet die benötigte Anzahl an Winkel auszuwählen. In Abbildung \ref{lem_faa_choice_ex} führt dies auf der rechten Seite zu keinem FAA und somit auch zu keiner SLTR. Die Konstruktion des Netzwerkes im Beweis von Proposition \ref{lem_faa} ist somit sinnvoll.

\begin{figure}[h]
	\centering
  	\includegraphics[width=1\textwidth]{lem_faa_choice_ex.png}
  	\caption{Bei der Auswahl der Winkel aus W ist Vorsicht geboten.}
	\label{lem_faa_choice_ex}
\end{figure}
\end{example}

\subsection{Minimale Schnitte in $\mathcal{N}_G$}

Wir wollen in diesem Abschnitt einen ersten möglichen Beweisansatz von Vermutung \ref{int_conj} besprechen. Die Idee ist, dass wir unter der Annahme das nur eine nicht-ganzzahlige Lösung existiert einen minimalen Schnitt in einem Teilnetzwerk von $\mathcal{N}_G$ erzeugen würden, und so zu einem Widerspruch gelangen. 

Angenommen, es existiert ein Netzwerk $\mathcal{N}_G$, für das nur eine nicht ganzzahlige Lösung existiert. Sei $\tilde{\varphi}$ dieser nicht ganzzahlige zulässige Fluss und $\phi$ ein wie in Proposition \ref{lem_faa} aus $\tilde{\varphi}$ konstruiertes FAA für $G$. Sei $\varphi_z$ der eindeutige Zuweisungs-Fluss der dieses FAA auf $\mathcal{N}_G$ kodiert und $\overline{\mathcal{N}}_G$, ein Teilnetzwerk von $\mathcal{N}_G$, aus welchem alle Kanten, die von $\varphi_z$ ausgelastet sind, gelöscht wurden. Die Bedarfe sind weiterhin $|E_{in}|$ und $3|F_{in}|$ für den Schnyder- und Ecken-Fluss. Nach Proposition \ref{choose_types} können wir $\varphi_s$ und $\varphi_e$ zusammenfassen und mit $\varphi_1$ bezeichnen. Wir suchen also nach einem zulässigem ganzzahligem Fluss $\varphi_1 = \varphi_s + \varphi_e$ auf $\overline{\mathcal{N}}_G$ mit Bedarf $|E_{in}| + 3|F_{in}|$, da dann auch eine ganzzahlige Lösung $(\varphi_s,\varphi_e)$ folgen würde. Wir hätten somit einen ganzzahligen Fluss auf $\mathcal{N}_G$ konstruiert, was zu einem Widerspruch führt.

Nach dem Max-Flow Min-Cut Theorem existiert ein zulässiger Fluss auf $\overline{\mathcal{N}}_G$ genau dann, wenn es keinen (Kanten-)Schnitt in $\overline{\mathcal{N}}_G$ mit Kapazität kleiner als $|E_{in}| + 3|F_{in}|$ gibt. Bevor wir fortfahren wollen wir einige Kantentypen aus $\mathcal{N}_G$ benennen.

\begin{itemize}
\item $E_\triangle = $ Die äußeren Kanten in den Winkeldreiecken.
\item $E_\triangledown = $ Die inneren Kanten in den Winkeldreiecken.
\item $S_* =$ Die Kanten von den Dummy-Knoten zur Dummy-Senke.
\item $V_* = $ Die Kanten von den Winkeldreiecken zu den Dummy-Knoten.
\item $E_{\to} = $ Die Kanten von Quelle 1 zu den Kanten-Knoten.
\item $F_\square = $ Die Kanten von den kleinen Quadraten zu inneren Gebieten $f$.
\item $V_{\to} = $ Die Kanten von den Knoten-Knoten zu Senke 1.
\item $e_{d} = $ Die Kante von der Dummy-Senke zu Senke 2
\end{itemize}

Sowohl $\mathcal{S}_1 = E_\triangle \cup E_{\to}$, als auch $\mathcal{S}_2 = F_\square \cup V_{\to} \cup \{e_{d}\}$ sind minimale Schnitte in $\mathcal{N}_G$. Für beide Menge gilt $|\mathcal{S}_1| = |\mathcal{S}_2| = |E_{in}| + \sum{f \in F_{in}}$ und sie trennen die Quellen ($\mathcal{S}_1$) bzw. die Senken ($\mathcal{S}_2$) vom Rest des Netzwerkes ab. Wenn wir nur von den Kanten aus $E_\triangle$, die in $\overline{\mathcal{N}}_G$ übrig sind, sprechen, schreiben wir $\overline{E}_\triangle$. Für die, zu diesen korrespondierenden Kanten im inneren ihrer Winkeldreiecke, schreiben wir $\overline{E}_\triangledown$. Für die Teilmengen von $V_*$ und $S_*$ in $\overline{\mathcal{N}}_G$ schreiben wir $\overline{S}_*$ und $\overline{V}_*$. Die restlichen Mengen sind vollständig in $\overline{\mathcal{N}}_G$ enthalten.

Seien $E_z$ die von $\varphi_z$ genutzen Kanten, die wir aus $\mathcal{N}_G$ entfernen. Dann folgt $|\mathcal{S}_1 \cap E_z| = |E_\triangle \cap E_z| = |\varphi_z|$. Somit ist $\overline{\mathcal{S}}_1 = \mathcal{S}_1 \backslash E_z = \overline{E}_\triangle \cup E_\to$ ein Schnitt in $\overline{\mathcal{N}}_G$. Analog ist $\overline{\mathcal{S}}_2 = F_\square \cup V_{\to}$ ein Schnitt. Für die Kapazität von $\overline{\mathcal{S}}_1$ können wir folgern 
$$ c(\overline{\mathcal{S}}_1) = c(\overline{E}_\triangle) + c(E_\to) = c(E_\triangle) - |\varphi_z| + c(E_\to) = 3|F_{in}| + |E_{in}|,$$
und wieder folgt analog $c(\overline{\mathcal{S}}_2) = 3|F_{in}| + |E_{in}|$.

Falls es sich hierbei um minimale Schnitte handelt, dann würde dies bedeuten, dass eine ganzzahlige Lösung für $\overline{\mathcal{N}}_G$ existiert, mit deren Hilfe wir, zusammen mit $\varphi_z$, eine ganzzahlige zulässige Lösung für $\mathcal{N}_G$ konstruieren könnten, was wiederum ein Widerspruch zu unserer Annahme wäre. Es muss also einen kleineren Schnitt $\mathcal{S}_{min}$, mit $|\mathcal{S}_{min}| \leq 3|F_{in}| + |E_{in}| - 1$, geben. 

\begin{remark}
Falls wir nach dem selben Schema die Kanten aus $\mathcal{N}_G$ entfernen, welche von einem Zuweisungs-Fluss gesättigt sind, der einem FAA entspricht, das keine SLTR induziert, dann muss so ein minimaler Schnitt $\mathcal{S}_{min}$ existieren. Sonst würde ein Widerspruch zu Theorem \ref{theo_ccc} entstehen.
\end{remark}

Falls Vermutung \ref{int_conj} stimmt, dann existiert ist dies nicht zwangsläufig $\mathcal{S}_{min}$ und falls Vermutung \ref{conj_faa} Korrekt ist, dann kann so ein Schnitt nicht existieren. Nehmen wir jedoch für den Moment an, das $\mathcal{S}_{min}$ wie oben beschrieben existiert, dann können wir die folgenden Beobachtungen festhalten.

\begin{claim} \label{cut_types1}
Falls $\mathcal{S}_{min}$ existiert, dann muss auch ein minimaler Schnitt $\mathcal{S}_{min}^*$ in $\overline{\mathcal{N}}_G$ existieren, sodass er nur Kanten von einem der vier Typen $\overline{E}_\triangledown, F_\square, V_\to$ und $E_\to$ enthält.
\end{claim}

\begin{figure}
	\centering
  	\includegraphics[width=0.8\textwidth]{face_cut.png}
  	\caption{Die vier Kantentypen $\overline{E}_\triangledown$ (rot), $F_\square$ (blau), $V_\to$ (orange) und $E_\to$ (grün) aus denen sich, nach Behauptungen \ref{cut_types1} und \ref{cut_types2}, ein minimaler Schnitt in $\overline{\mathcal{N}}_G$ zusammensetzen müsste.}
	\label{cut_edges}
\end{figure}

Betrachten wir die Kanten in $\mathcal{S}_{min}$. Die vier Kantentypen sind in Abbildung \ref{cut_edges} eingezeichnet. auf einem Pfad von der Quelle bis zu einer Kante in $\overline{E}_\triangledown$, können in $\mathcal{S}_{min}^*$ durch diese ersetzt werden. Ebenso können Kanten zwischen zwei Winkeldreiecken, oder von einem Winkeldreieck zu einem kleinen Quadrat in $\mathcal{S}_{min}^*$ durch die entgegen dem Uhrzeigersinn nächste Kante in $\overline{E}_\triangledown$ ersetzt werden. Kanten zwischen einem Kanten-Knoten und einem Knoten-Knoten, oder einem kleinen Quadrat, können in $\mathcal{S}_{min}^*$ durch eine Kante in $E_\to$ ersetzt werden. Abschliessend können Kanten, von einem inneren Gebiet zu Senke in $\mathcal{S}_{min}^*$ ,durch das hinzufügen von allen Kanten aus $F_\square$ an diesem Gebiet, ersetzt werden.

\begin{claim}\label{cut_types2}
Falls $\mathcal{S}_{min}$ existiert, dann muss auch ein minimaler Schnitt $\mathcal{S}_{min}^*$ in $\overline{\mathcal{N}}_G$ existieren, sodass er nur Kanten von einem der vier Typen $\overline{E}_\triangledown, F_\square, V_\to$ und $E_\to$ enthält. Er enthält aus jeder der Mengen $\overline{E}_\triangledown, F_\square, V_\to$ und $E_\to$ mindestens eine, aber aus keiner der Mengen alle Kanten.
\end{claim}

\begin{proof}
Falls ein solcher ein Schnitt $\mathcal{S}_{min}^*$ existiert, dann kann $\mathcal{S}_{min}^*$ nicht alle Kanten $\overline{E}_\triangledown$ enthalten. Sonst könnten wir aus $\mathcal{S}^*_{min} \cup (E_\triangle \cap E_z)$ einen Schnitt $\mathcal{S}$ mit der gleichen Kapazität konstruieren, indem wir die Kanten $E_\triangledown \cap \mathcal{S}^*_{min}$ durch die Korrespondierenden Kanten in $E_\triangle$ ersetzen. Es folgt $\mathcal{S} \supseteq\mathcal{S}_1$, was ein Widerspruch ist. Falls $\mathcal{S}_{min}^*$ jedoch keine Kante aus $\overline{E}_\triangledown$ enthält, dann muss $\mathcal{S}_{min}^*$ alle Kanten aus $F_\square$ enthalten, weil $\mathcal{S}_{min}^*$ ein Schnitt ist. Falls $\mathcal{S}_{min}^*$ alle Kanten aus $F_\square$ enthält, dann können wir annehmen, dass $\mathcal{S}_{min}^*$ auch alle Kanten aus $V_\to$ enthält. Es folgt $\mathcal{S}^*_{min} \cup \{e_d\} \supseteq \mathcal{S}_2$, was erneut einen Widerspruch bedeutet. Angenommen er enthält keine Kante aus $F_\square$, dann muss er alle Kanten aus $\overline{E}_\triangledown$ und $E_\to$ enthalten und es würde $\mathcal{S}^*_{min} \cup (E_\triangle \cap E_z)$ wäre wie oben erneut ein Schnitt mit Kapazität $\geq |\mathcal{S}_1|$. 

Es bleibt die Mengen $E_\to$ und $V_\to$ zu betrachten. Aus $\mathcal{S}_{min}^*\cap E_\to = \emptyset $ folgt $F_\square\subseteq\mathcal{S}_{min}^*$ und aus $\mathcal{S}_{min}^*\cap V_\to = \emptyset $ folgt $E_\to \subset  \mathcal{S}_{min}^*$. Im Fall $E_\to \subset  \mathcal{S}_{min}^*$ brauchen wir in jedem inneren Gebiet mindestens drei Kanten in $\mathcal{S}_{min}^*$, womit wir wieder mindestens Kardinalität $|S_1|$ erreichen. Wir betrachten als letztens $V_\to$. Hierbei ist zu beachten, dass für Knoten $v$ mit deg$(v) \leq 3$ keine Kante in $\mathcal{N}_G$ existiert. Es gelte $V_\to \subset  \mathcal{S}_{min}^*$, dann muss $\mathcal{N}_G$ noch mindesten $|E_\to|-|V_\to|$ Kanten enthalten, um den Schnyder-Fluss zu unterbrechen, was ein Widerspruch ist. Gelte nun $\mathcal{S}_{min}^*\cap V_\to = \emptyset$. Wir benötigen erneut mindestens $|E_{in}|$ Kanten in $\mathcal{S}_{min}^*$ um den Schnyder-Fluss zu unterbrechen und erhalten somit einen letzten Widerspruch. Behauptung \ref{cut_types2} ist somit richtig.
\end{proof}

Schnyder- und Ecken-Fluss in $\overline{\mathcal{N}}_G$ können nur über die Kanten von Typ $E_\to, V_\to$ und $F_\square$ beziehungsweise $\overline{E}_\triangledown$ und $F_\square$ fließen. Der einzige Kantentyp der in beiden vorkommt ist $F_\square$. Wir finden genau dann keinen ganzzahligen Fluss auf $\overline{\mathcal{N}}_G$, wenn es keinen Ecken-Kompatiblen Schnyder-Wood $\sigma$ zu $\phi$ gibt. Somit muss jeder zulässige Schnyder-Fluss in mindestens einem Gebiet die kleinen Quadrate so auslasten, dass es aus mindestens einem freien Winkel keinen Ecken-Pfad geben kann. Alle kleinen Quadrate zwischen einem freien Winkel und dem im Uhrzeigersinn nächsten müssen somit vom Schnyder-Fluss gesättigt sein. So ein kleines Quadrat ist in Abbildung \ref{combined_face_not_corner}) rot eingefärbt. Wir nennen die Kante an einem kleinen Quadrat ein \textit{blockierendes} Quadrat, falls ein zulässiger Schnyder-Fluss existiert, sodass an diesem kleinen Quadrat ein Widerspruch zur Ecken Kompatibilität auftritt.

\begin{claim}
Falls $\mathcal{S}_{min}$ existiert, dann existiert ein Schnitt $\mathcal{S}^*_{min}$ wie in Behauptung \ref{cut_types2}, welcher alle blockierenden Quadrate enthält.
\end{claim}

Falls ein blockierenden Quadrat nicht in $\mathcal{S}^*_{min}$ enthalten ist, enthält $\mathcal{S}^*_{min}$ die korrespondierenden Kanten aus $E_\to$ und die gegen den Uhrzeigersinn nächste Kante aus $E_\triangledown$. Da es sich um ein blockierendes Quadrat handelt kann der Ecken-Fluss aus diesem Winkeldreieck nicht vorher das Gebiet verlassen. Wir können somit die Kante am Winkeldreieck in $\mathcal{S}^*_{min}$ durch das blockierende Quadrat ersetzen.

\begin{figure}
	\centering
  	\includegraphics[width=0.8\textwidth]{combined_face_not_corner.pdf}
  	\caption{Der Schnyder-Fluss (rot) ist nicht Ecken kompatibel zum FAA $\phi$ (grün). Somit muss ein blockiertes Quadrat (hier in rot) in $\mathcal{N}_G$ existieren und durch mindestens einen freien Winkel kann kein Ecken-Pfad führen.}
	\label{combined_face_not_corner}
\end{figure}



$$|\mathcal{S}_{min}^*\cap\{E_\to \cup V_\to \cup F_\square \}| \geq |\tilde{\varphi_z}| \text{  und  } |\mathcal{S}_{min}^*\cap\{\overline{E}_\triangledown \cup F_\square \}| \geq |\tilde{\varphi_e}|.$$



