\chapter{Algorithmen zur Erkennung von SLTR}

Im vorherigen Kapitel wurden Kriterien für die Existenz einer SLTR erarbeitet, die allerdings nicht sofort einen Algorithmus, sowohl zur Frage nach der Existenz, als auch zum erlangen einer spezifischen SLTR liefern. Diesem Thema wollen wir uns nun im folgenden Kapitel zuwenden und dafür zum Einstieg einen von Aerts und Felsner in \cite{af13} erarbeiteten Algorithmus betrachten.

\section{SLTR via Zwei-Fluss}

Wir betrachten im folgenden gerichtete Graphen. Das Ziel ist es, für einen gegebenen Graphen sowohl einen Schnyder Wood als auch ein FAA jeweils als Lösung eines Fluss-Problems zu erhalten und diese beiden dann in einem Zwei-Fluss-Problem zu kombinieren, sodass eine Lösung ein Ecken-Kompatibles-Paar gibt und wir somit eine SLTR erhalten. Wir beschäftigen uns also mit der folgenden Problemstellung.

\begin{definition}[Gerichtetes-Multi-Fluss-Problem]
Sei $D=(V,E)$ ein gerichteter Graph, im Weiteren auch Netzwerk genannt, mit den Kapaziäten $c:E\mapsto\mathbb{R}_{+}$, Paaren von ausgezeichneten Knoten $\{(s_1,t_1), ... ,(s_n,t_n)\}$ und positiven Bedarfen $\{d_1, ... ,d_n\}$, dann ist $\varphi=(\varphi_1, ... ,\varphi_n)$ ein zulässiger Fluss, falls
\begin{itemize}
\item[F1] $\forall (u,v) \in E : \sum_{i=1}^{n}{\varphi_i(u,v)} \leq c(u,v) $
\item[F2] $ \forall u \neq s_i,t_i : \sum_{w \in V} \varphi_i(u,w) - \sum_{w \in V} \varphi_i(w,u) $
\item[F3] $ \forall s_i : \sum_{w \in V} \varphi_i(s_i,w) - \sum_{w \in V} \varphi_i(w,s_i) = d_i $
\item[F4] $ \forall t_i : \sum_{w \in V} \varphi_i(w,s_i) - \sum_{w \in V} \varphi_i(s_i,w) = d_i $
\end{itemize}
\end{definition}

Im Fall $n=1$ und Kapazitäten $c:E\mapsto\mathbb{N}$ impliziert die Existenz eines zulässigen Flusses die Existenz einer ganzzahligen Lösung. Für $n=2$ und wir ungerichtete Netzwerke zeigt \cite{hu} die selbe Eigenschaft. Dies gilt jedoch im Fall $n \geq 2$ für gerichtete Graphen nicht mehr.

\subsection{Schnyder Wald Fluss}

Um einen Schnyder Wood zu erhalten folgen wir \cite{felsner04lattice}.\
Wir betrachten den Primal-Dual Graphen $G+G^*$ eines planen Graphen $G$. Hier ist $G^*$ der schwache duale Graph zusammen mit einer Halbkante ins äussere Gebiet von jeder inzidenten Kante aus. Die Menge der Knoten von $G+G^*$ besteht aus Knoten-Knoten, Kanten-Knoten und Gebiets-Knoten, mit Kanten in $G+G^*$, sowohl zwischen inzidenten Kanten und Knoten, als auch Kanten und Gebieten in $G$. Somit ist $G+G^*$ bipartit. Falls wir einen Knoten $f_\infty$ für das äussere Gebiet einsetzten und die Halbkanten verlängern sprechen wir vom Abschluss von $G+G^*$. Wir bezeichnen diesen mit $\tilde{G}$.\\
Sei $G=(V,E)$ ein Graph und $\alpha:V\mapsto\mathbb{N}$ eine Funktion auf $G$. Eine $\alpha-Orientierung$ ist eine Orientierung auf $G$, sodass der Ausgrad eines jeden Knoten $\alpha(v)$ entspricht. Das folgende Theorem stammt ebenfalls aus \cite{felsner04lattice}.

\begin{theorem}
Sei $G$ ein planer Graph mit Aufhängungen $\{a_1,a_2,a_3\}$, dann sind die folgenden Strukturen in Bijektion:
\begin{itemize}
\item Die Schnyder Wälder auf $G$.
\item Die Schnyder Wälder auf dem (schwachen) dualen Graph $G^*$.
\item Die $\alpha_{s}-Orientierungen$ des Abschlusses von $G+G^+$ mit $\alpha_s(v) = 3$ für jeden Knoten- und Gebiets-Knoten,  $\alpha_s(e) = 1$ für jeden Kanten-Knoten und  $\alpha_s(f_\infty) = 0$.
\end{itemize}
\end{theorem}

Fusy zeigt in \cite{fusy07} im Zuge der Untersuchung spezifischer $\alpha$-Funktionen, dass sich $\alpha_s$-Orientierungen von $G+G^*$ in linearer Zeit berechnen lassen.\\
%%  TODO Do i need this??

Machen wir uns also an die Konstruktion eines Netzwerks $\mathcal{N}$ mit einer Quelle und Senke, sodass eine zulässige Lösung $\varphi$ einer $\alpha_s-Orientierung$ von $\tilde{G}$ entspricht, und somit auch einen Schnyder Wald auf $G$ liefert. Besonderes Augenmerk ist hier auf die Möglichkeit einer späteren Kombination mit einem FAA Fluss gelegt, um ein Zwei-Fluss-Problem zu erstellen, und nicht unbedingt auf Effizienz.\

Wie oben schon erwähnt ist $\tilde{G}$ bipartit, Kanten-Knoten haben Grad 4, Knoten-Knoten Grad $deg(v)$ und Gebiets-Knoten Grad $deg(f)$. Für eine $\alpha_s$-Orientierung muss jeder Kanten-Knoten Ausgrad 1, jeder Knoten-Knoten Eingrad $deg(v)-3$ und jeder Gebiets-Knoten Eingrad $deg(f)-3$ haben. Die Kanten-Knoten am äusseren Gebiet sind in $\tilde{G}$ immer nach aussen orientiert. Somit müssen wir nur die inneren Kanten-Kanten $E_{in}$ betrachten. \

Sei $\mathcal{N}$ ein Netzwerk mit jeweils einer Quelle $s$ und Senke $t$, Kanten von der Quelle zu jedem $e \in E_{in}$ mit Kapazität 1, Kanten von den Kanten-Knoten $e$ zu inzidenten Knoten-Knoten $v$ und (inneren) Gebiets-Knoten $f \in F_{in}$ in $G$ ebenfalls mit Kapazität 1, Kanten von $f \in F_{in}$ zur Senke mit Kapazitäten $deg(f)-3$, Kanten von den (inneren) Knoten-Knoten $v \in V_{in} = V \setminus \{a_1,a_2,a_3\}$ zur Senke mit Kapazitäten $deg(v)-3$ und Kanten von den Aufhängungen $a_i$ zur Senke mit Kapazitäten $deg(v)-2$. Die letzte Kapazität resultiert aus dem Fakt, dass die Halbkante in $G+G^*$ immer nach aussen orientiert ist und wir somit nur noch zwei andere Kanten nach aussen orientieren müssen.

%% Bild!!

Der Bedarf des Netzwerkes entspricht der Anzahl der inneren Kanten von $G$. Sei $\varphi$ eine zulässige ganzzahlige Lösung, dann hat jeder Kanten-Knoten $e$ Ausgrad 1. Der Fluss entlang einer Auskante von $e \in E_{in}$ in $\mathcal{N}$ entspricht dann genau der hin zu $e$ orientierten Kante in $G+G^*$. Somit entspricht $\varphi$ einem Schnyder Wald auf $G$.

\subsection{FAA Fluss}

Um ein FAA auf $G$ mit gegebenen Aufhängungen $\{a_1,a_2,a_3\}$ zu erhalten müssen wir jedem inneren Gebiet $f \in F_{in}$ genau drei Ecken und $deg(f)-3$ flache Winkel zuordnen und jeder innere Knoten darf maximal einem Gebiet zugeordnet, also in diesem flach, sein.\


Sei also wieder $\mathcal{N}$ ein Netzwerk mit einer Quelle und Senke, einem Knoten für jeden inneren Winkel $(f,v)$ für $v\in V, f \in F_{in}$, und Knoten für alle inneren Gebiete $f$ und alle Knoten $v$. Von der Quelle existiert eine Kante mit Kapazität 1 zu jedem inneren Winkel $(f,v)$, von jedem inneren Winkel $(f,v)$ jeweils eine Kante zu $f$ und zu $v$ mit Kapazität 1, von jedem inneren Gebiet $f$ eine Kante mit Kapazität 3 zur Senke und zuletzt noch eine Kante von jedem inneren Knoten $v$ zur Senke mit Kapazität 1.

%% Bild!!

Der Bedarf des Netzwerks ist $\sum_{f \in F_{in}}{deg{f}}, die Anzahl der inneren Winkel von $G$. Sei $\varphi$ eine zulässige ganzzahlige Lösung, dann entspricht Fluss auf einer Kante ((f,v),f) einer Ecke von f und Fluss auf ((f,v),u) einem flachen Winkel. Die gegebenen Kapazitäten gewährleisten genau drei Ecken für jedes Gebiet und maximal einen flachen Winkel an jedem inneren Knoten.


