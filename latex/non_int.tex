\chapter{Nicht ganzzahlige Flüsse}

Wie in Kapitel \ref{main_algo} erwähnt, impliziert eine nicht ganzzahlige Lösung für ein Multi-Fluss-Problem auf einem gerichteten Graphen mit $n\geq 2$ Paaren von Quellen und Senken, im Allgemeinen nicht die Existenz einer ganzzahligen Lösung. Die Ergebnisse aus Kapitel \ref{the_program}, lassen jedoch die Möglichkeit offen, dass das man in unserem Fall die folgende Aussage beweisen kann.

\begin{conjecture}\label{int_conj}
Sei $\tilde{\varphi}=(\tilde{\varphi_1},\tilde{\varphi_2})$ ein nicht ganzzahliger zulässiger Fluss auf $\mathcal{N}_G$, dann existiert auch eine ganzzahlige Lösung $\varphi$ und wir können in konstanter Zeit ein Gutes-FAA aus $\tilde{\varphi_2}$ extrahieren, ohne eine ganzzahlige Lösung zu berechnen.
\end{conjecture}

\begin{remark}
Wenn wir nicht darauf bestehen, dass unsere Lösung ganzzahlig ist, dann lässt sich eine Lösung nach TODO durch lineare Programmierung in polynomineller Zeit finden und das Entscheidungsproblem, ob ein Graph eine SLTR hat läge so in $\mathcal{P}$.
\end{remark}

Um die Argumentation einfacher zu gestalten, werden wir unser 2-Fluss Problem manchmal als 3-Fluss Problem mit einer Lösung $\varphi=(\varphi_s,\varphi_e,\varphi_z)$ behandeln. Wir erstellen also $\mathcal{N}_G^*$, wie zuvor $\mathcal{N}_G$, nur mit drei Quellen und Senken und weisen Schnyder-, Zuweisungs- und Ecken-Fluss eigene Typen zu. Man kann leicht sehen, dass Theorem \ref{theo_algo} in angepasster Form hier ebenfalls gilt und ein zulässiger  Fluss $(\varphi_s,\varphi_e,\varphi_z)$ auf $\mathcal{N}_G^*$ genau dann existiert, wenn auch ein zulässiger Fluss $(\varphi_1,\varphi_2)$ auf $\mathcal{N}_G$ existiert. Die Hinrichtung ist klar. Nehmen wir an $(\varphi_1,\varphi_2)$ ist eine ganzzahlige Lösung. Nach Beobachtung A2 gilt, dass die äusseren Kanten eines Winkel-Dreiecks entweder von einem Ecken- oder einem Zuweisungs-Pfad genutzt werden. Diese Kanten sind, zusammen mit den Kanten von Quelle 2 zu den Winkeldreiecken, die einzigen in $\mathcal{N}_G$ die von beiden Flüssen genutzt werden. Wir können also $\varphi_2$ in $|\varphi_2|$ ganzzahlige Pfade aufteilen und jeden Pfad entweder $\varphi_e$ oder $\varphi_z$ zuweisen, je nachdem ob er über die Dummy-Senke führt, oder nicht. Insbesondere folgt mit der gleichen Argumentation:  

\begin{itemize}
\item [O1] Jede beliebige Kombination von $\varphi_s,\varphi_e$ und $\varphi_z$ zu zwei Flüssen, führt zu einem zu unserem Ansatz analogen Netzwerk $\mathcal{N}_G$, und eine zulässige ganzzahlige Lösung existiert entweder für alle, oder für keines.
\end{itemize}

Betrachten wir zunächst den zweiten Teil von Vermutung \ref{int_conj}.

\begin{lemma}\label{lem_faa}
Sei $\tilde{\varphi}$ ein nicht ganzzahliger zulässiger Fluss auf $\mathcal{N}_G$ und sei $W$ die Menge der vom Zuweisungsfluss $\varphi_z$ genutzten inneren Winkel von $G$. Dann entspricht jede Teilmenge $W_{F}\subseteq W$ mit $|W_F| = \sum_{f \in F_{in}}(|f|-3)$ in der jeder Knoten höchstens einmal vorkommt einem FAA von $G$.
\end{lemma}

\begin{proof}
Da $\tilde{\varphi}$ ein zulässiger Fluss ist, muss $|\tilde{\varphi}_z| = \sum_{f \in F_{in}}(|f|-3)$ gelten. Seien $P_z$ die (nicht ganzzahligen) Pfade die zusammen $\tilde{\varphi}_z$ ergeben, dann gilt $|\tilde{\varphi}_z(p)| \in (0,1]$ für $p \in P_z$, wobei $|\tilde{\varphi}_z(p)|$ die Menge des Zuweisungsflusses entlang $p$ beschreibt. Es gilt $|P_z| \geq \sum_{f \in F_{in}}(|f|-3)$ und ebenso $|W| \geq \sum_{f \in F_{in}}(|f|-3)$.

Sei $\phi$ die Menge der zugewiesenen Winkel $(f_{aus},v)$ im äusseren Gebiet. Fügen wir nun einen beliebigen Winkel $(f,v)\in W$, zu $\phi$ hinzu, löschen alle Winkel $W_v$ aus $W$ die $v$ enthalten und alle Pfade $P_v$ aus $P$ die durch $v^*$ laufen. Dann gilt $|\varphi_z(P_v)| \leq 1$ und somit $|\varphi_z(P\backslash P_v)| \geq (\sum_{f \in F_{in}}(|f|-3)) - 1$. Für $W \backslash W_v$ folgt ebenfalls $|W \backslash W_v| \geq (\sum_{f \in F_{in}}(|f|-3))- 1$, da an jedem anderen von $\varphi_z$ genutzten $v^*$ noch mindestens ein Winkel aus $W\backslash W_v$ liegt. Wir können diesen Schritt somit insgesamt $\sum_{f \in F_{in}}(|f|-3)$ mal durchführen und jeder Knoten ist nur in einem Winkel aus $\phi$ enthalten. Es folgt $|\phi| = \sum_{f \in F_{in}}(|f|-3) + |f_{aus}| - 3 = \sum_{f \in F}(|f|-3)$. Die so erhaltene Menge $\phi$ entspricht also einem FAA von $G$.
\end{proof}

Wenn wir zeigen können, dass die so erhaltenen $\phi$ Gute-FAAs sind, folgt Vermutung \ref{int_conj}, da die Existenz eines Guten-FAAs $\phi$ nach Theorem \ref{theo_algo} auch die Existenz eines ganzzahligen zulässigen Flusses $\varphi$ für $\mathcal{N}_G$ impliziert.

Nehmen wir an wir haben einen Graphen $G$ gefunden für den nur eine nicht ganzzahlige Lösung existiert. Sei $\tilde{\varphi}$ dieser nicht ganzzahlige zulässige Fluss auf $\mathcal{N}_G$, und $\phi$ ein wie in Lemma \ref{lem_faa} aus $\tilde{\varphi}$ konstruiertes FAA für $G$. Sei $\overline{\varphi_z}$ ein Zuweisungs-Fluss der dieses FAA auf $\mathcal{N}_G$ kodiert. Betrachte nun $\overline{\mathcal{N}_G}$, ein Teilnetzwerk von $\mathcal{N}_G$, aus welchem alle Kanten, die von $\overline{\varphi_z}$ genutzt werden gelöscht wurden. Wir wollen nun analysieren unter welchen Bedingungen eine ganzzahlige Lösung $(\varphi_s,\varphi_e)$ auf $\overline{\mathcal{N}_G}$ existiert.

Nach der in O1 festgehaltenen Beobachtung können wir, wie in \cite{af15}, $\varphi_s$ und $\varphi_e$ zusammenfassen und als $\varphi_1$ betrachten. Nach Theorem \ref{theo_int_flow} impliziert eine nicht ganzzahlige auch eine ganzzahlige Lösung $\varphi_1=\varphi_s+\varphi_e$. Und eine solche Lösung ist nach \textit{Max-Flow Min-Cut} maximal, genau dann, wenn ein minimaler Schnitt mit der entsprechenden Kapazität existiert. Da keine ganzzahlige Lösung existiert, kann es also keinen minimalen Schnitt $\mathcal{S} \subset E_{\overline{\mathcal{N}_G}}$ in $\overline{\mathcal{N}_G}$, mit $c(\mathcal{S}) = |\tilde{\varphi_s}+\tilde{\varphi_e}| = |E_{in}| + 3|F_{in}|$, geben, da wir sonst zu einem Widerspruch gelangen.\\

Bevor wir fortfahren wollen wir ein paar Kantentypen aus $\mathcal{N}_G$ benennen, und einige Beobachtungen zu ihnen festhalten.

\begin{itemize}
\item $E_\triangle = $ Die äusseren Kanten in den Winkeldreiecken.
\item $E_\triangledown= $ Die inneren Kanten in den Winkeldreiecken.
\item $F_\square = $ Die Kanten von den kleinen Quadraten zu inneren Gebieten $f$.
\item $V_{*} = $ Die Kanten von den Knoten-Knoten zu Senke 1.
\item $E_{*} = $ Die Kanten von Quelle 1 zu den Kanten-Knoten.
\end{itemize}

Wenn wir nur von den Kanten aus $E_\triangle$, die in $\overline{\mathcal{N}_G}$ übrig sind, sprechen, schreiben wir $\overline{E_\triangle}$. Für die, zu diesen korrespondierenden Kanten im inneren dieser Winkeldreiecke, schreiben wir $\overline{E_\triangledown}$. Sei $e_{d}$ die Kante von der Dummy-Senke zu Senke 2, dann sind sowohl $\mathcal{S}_1 = E_\triangle \cup E_{*}$, als auch $\mathcal{S}_2 = F_\square \cup V_{*} \cup \{e_{d}\}$ minimale Schnitte in $\mathcal{N}_G$.

Sei $E_z$ die Menge der Kanten, die wir aus $\mathcal{N}_G$ entfernen. Dann folgt $|\mathcal{S}_1 \cap E_z| = |E_\triangle \cap E_z| = |\varphi_z|$ und $\overline{\mathcal{S}_1} = \mathcal{S}_1 \backslash E_z = \overline{E_\triangle} \cup E_*$ ist ein Schnitt in $\overline{\mathcal{N}_G}$. Für die Kapazität dieses Schnittes können wir folgern 
$$ c(\overline{\mathcal{S}_1}) = c(\overline{E_\triangle}) + c(E_*) = c(E_\triangle) - |\varphi_z| + c(E_*) = 3|F_{in}| + |E_{in}|.$$

Falls es sich hierbei um einen minimalen Schnitt handelt, dann würde dies bedeuten, dass eine ganzzahlige Lösung für $\overline{\mathcal{N}_G}$ existiert, mit deren Hilfe wir, zusammen mit $\varphi_z$, eine ganzzahlige zulässige Lösung für $\mathcal{N}_G$ konstruieren könnten, was wiederum ein Widerspruch zu unserer Annahme wäre. Es muss also einen kleineren Schnitt $\mathcal{S}_{min}$, mit $|\mathcal{S}_{min}| < |\overline{\mathcal{S}_1}|$, geben. Wir nehmen für die folgende Aussage  ohne Beschränkung der Allgemeinheit an, dass wir in einem Schnitt $\mathcal{S}_{min}$ Kanten aus $E_\triangle$ mit den korrespondierenden Kanten aus $E_\triangledown$ ersetzten können.

\begin{claim}

Der minimale Schnitt $\mathcal{S}_{min}$ in $\mathcal{N}_G$ muss aus jeder der Mengen $\overline{E_\triangledown}, F_\square, V_*$ und $E_*$ mindestens eine, aber aus keiner der Mengen alle Kanten enthalten.

\end{claim}

Angenommen  TODO




