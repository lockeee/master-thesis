\chapter{Fazit}

Wir haben uns mit der Charakterisierung von geradlinigen Dreiecks-Darstellungen (SLTRs) beschäftigt und in Kapitel \ref{main_theory} sowohl notwendige als auch hinreichende Bedingungen für ihre Existenz erarbeitet und uns dabei an Aerts und Felsner orientiert \cite{af15}.

Es wird ein Zusammenhang zwischen SLTRs und Ecken kompatiblen Paaren aus einem Schnyder Wood und einem FAA auf ebenen Graphen bewiesen. Auf diesem Ergebnis aufbauend, haben wir in Kapitel \ref{main_algo} ein Netzwerk zur Lösung eines Gerichteten-Multi-Fluss-Problems konstruiert, auf dem ein zulässiger ganzzahliger Fluss einer SLTR entspricht. In Kapitel \ref{the_program} haben wir mithilfe dieses Netzwerks einen Algorithmus entwickelt, der die maschinelle Berechnung von SLTRs ermöglicht.

Aerts und Felsner zeigten, dass ein FAA ein System aus harmonischen Gleichungen induziert, dessen Lösung in Abhängigkeit einer Gewichtsfunktion $\{\lambda_e\}_{e\in E}$ die Zeichnung einer SLTR ergibt \cite{af13}. Wir haben diesen Ansatz in Kapitel \ref{the_program} genutzt, um nach der Berechnung eines Guten-FAA eine Zeichnung zu erstellen und haben dabei unterschiedliche Ansätze bei der Wahl der Gewichtsfunktion $\{\lambda_e\}$ verfolgt.

Experimentellen Berechnungen, mit dem in Kaptitel \ref{the_program} beschriebenen Programm, deuten darauf hin, dass eine nicht-ganzzahlige Lösungen des Gerichteten-Multi-Fluss-Problems ausreicht, um eine SLTR zu bestimmen. In Abschnitt \ref{sec_non_int} haben wir uns deswegen mit nicht-ganzzahligen Lösungen des Fluss-Problems beschäftigt und einige dahingehende Beobachtungen gesammelt. Jedoch ist es uns nicht gelungen, die folgende Frage positiv zu beantworten: Liegt die Erkennung von SLTRs in P?