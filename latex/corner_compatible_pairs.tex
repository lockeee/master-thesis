\section{Ecken kompatible Paare}

In diesem Abschnitt werden wir uns mit einer zweiten Charakterisierung von SLTRs auf planaren Graphen nach \cite{af15} beschäftigen, die eine Verbindung zwischen Schnyder Woods und FAAs herstellt und so zu einer hinreichenden Bedingung für SLTRs führt. Wir beginnen wieder mit der Definition dieses Zusammenhanges.

\begin{definition}[Ecken Kompatibilität]\label{def_coco}
Ein Paar $(\sigma,\phi)$ aus einem Schnyder Labeling $\sigma$ und einem FAA $\phi$ nenne wir \textit{Ecken kompatibel}, falls:
\begin{itemize}
\item [C1] Das Schnyder Labeling $\sigma$ und das FAA $\phi$ nutzen die selben Aufhängungen.
\item [C2] In jedem inneren Gebiet haben die drei Ecken aus $\phi$ genau drei unterschiedliche Label in $\sigma$.
\end{itemize}
\end{definition}

TODO

\begin{theorem}\label{theo_coco}
Sei G ein planer, intern-3-zusammenhängender Graph mit Aufhängungen. G besitzt eine SLTR, genau dann wenn ein Ecken kompatibles Paar $(\sigma,\phi)$ aus einem Schnyder Labeling $\sigma$ und einem FAA $\phi$ existiert.
\end{theorem}

TODO